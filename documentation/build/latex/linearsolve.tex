%% Generated by Sphinx.
\def\sphinxdocclass{report}
\documentclass[letterpaper,10pt,openany,oneside,english]{sphinxmanual}
\ifdefined\pdfpxdimen
   \let\sphinxpxdimen\pdfpxdimen\else\newdimen\sphinxpxdimen
\fi \sphinxpxdimen=49336sp\relax

\usepackage[margin=1in,marginparwidth=0.5in]{geometry}
\usepackage[utf8]{inputenc}
\ifdefined\DeclareUnicodeCharacter
  \DeclareUnicodeCharacter{00A0}{\nobreakspace}
\fi
\usepackage{cmap}
\usepackage[T1]{fontenc}
\usepackage{amsmath,amssymb,amstext}
\usepackage[english]{babel}
\usepackage{times}
\usepackage[Bjarne]{fncychap}
\usepackage{longtable}
\usepackage{sphinx}

\usepackage{multirow}
\usepackage{eqparbox}

% Include hyperref last.
\usepackage{hyperref}
% Fix anchor placement for figures with captions.
\usepackage{hypcap}% it must be loaded after hyperref.
% Set up styles of URL: it should be placed after hyperref.
\urlstyle{same}

\addto\captionsenglish{\renewcommand{\figurename}{Fig.\@ }}
\addto\captionsenglish{\renewcommand{\tablename}{Table }}
\addto\captionsenglish{\renewcommand{\literalblockname}{Listing }}

\addto\extrasenglish{\def\pageautorefname{page}}

\setcounter{tocdepth}{1}


% Jupyter Notebook prompt colors
\definecolor{nbsphinxin}{HTML}{303F9F}
\definecolor{nbsphinxout}{HTML}{D84315}
% ANSI colors for output streams and traceback highlighting
\definecolor{ansi-black}{HTML}{3E424D}
\definecolor{ansi-black-intense}{HTML}{282C36}
\definecolor{ansi-red}{HTML}{E75C58}
\definecolor{ansi-red-intense}{HTML}{B22B31}
\definecolor{ansi-green}{HTML}{00A250}
\definecolor{ansi-green-intense}{HTML}{007427}
\definecolor{ansi-yellow}{HTML}{DDB62B}
\definecolor{ansi-yellow-intense}{HTML}{B27D12}
\definecolor{ansi-blue}{HTML}{208FFB}
\definecolor{ansi-blue-intense}{HTML}{0065CA}
\definecolor{ansi-magenta}{HTML}{D160C4}
\definecolor{ansi-magenta-intense}{HTML}{A03196}
\definecolor{ansi-cyan}{HTML}{60C6C8}
\definecolor{ansi-cyan-intense}{HTML}{258F8F}
\definecolor{ansi-white}{HTML}{C5C1B4}
\definecolor{ansi-white-intense}{HTML}{A1A6B2}

\usepackage{enumitem}
\setlistdepth{99}

\title{linearsolve Documentation}
\date{Feb 28, 2017}
\release{3.1.6}
\author{Brian C. Jenkins}
\newcommand{\sphinxlogo}{}
\renewcommand{\releasename}{Release}
\makeindex

\begin{document}

\maketitle
\sphinxtableofcontents
\phantomsection\label{\detokenize{index::doc}}


\sphinxcode{linearsolve} is a Python package for approximating, solving, and simulating dynamic stochastic general equilibrium (DSGE) models. \sphinxcode{linearsolve} is compatible with Python 2 and 3.


\chapter{Installation}
\label{\detokenize{index:installation}}\label{\detokenize{index:about-linearsolve}}
Install \sphinxcode{linearsolve} from PyPI with the shell command:

\begin{sphinxVerbatim}[commandchars=\\\{\}]
\PYG{n}{pip} \PYG{n}{install} \PYG{n}{linearsolve}
\end{sphinxVerbatim}


\chapter{Contents:}
\label{\detokenize{index:contents}}
\begin{Verbatim}[commandchars=\\\{\}]
\textcolor{nbsphinxin}{In [1]: }\PYG{k+kn}{import} \PYG{n+nn}{numpy} \PYG{k+kn}{as} \PYG{n+nn}{np}
        \PYG{k+kn}{import} \PYG{n+nn}{pandas} \PYG{k+kn}{as} \PYG{n+nn}{pd}
        \PYG{k+kn}{import} \PYG{n+nn}{matplotlib.pyplot} \PYG{k+kn}{as} \PYG{n+nn}{plt}
        \PYG{k+kn}{import} \PYG{n+nn}{linearsolve} \PYG{k+kn}{as} \PYG{n+nn}{ls}
        \PYG{n}{np}\PYG{o}{.}\PYG{n}{set\PYGZus{}printoptions}\PYG{p}{(}\PYG{n}{suppress}\PYG{o}{=}\PYG{n+nb+bp}{True}\PYG{p}{)}
        \PYG{o}{\PYGZpc{}}\PYG{k}{matplotlib} inline
\end{Verbatim}


\section{The \sphinxstyleliteralintitle{linearsolve.model} class}
\label{\detokenize{examples::doc}}\label{\detokenize{examples:The-linearsolve.model-class}}
The \sphinxcode{linearsolve.model} class package contains a several functions for
approximating, solving, and simulating dynamic stochastic general
equilibrium (DSGE) models. The equilibrium conditions for most DSGE
models can be expressed as a vector function \(F\):
\phantomsection\label{\detokenize{examples:equation-examples:0}}\label{equation:examples:examples:0}\begin{align}
f(E_t X_{t+1}, X_t, \epsilon_{t+1}) = 0,
\end{align}
where 0 is an \(n\times 1\) vector of zeros, \(X_t\) is an
\(n\times 1\) vector of endogenous variables, and
\(\epsilon_{t+1}\) is an \(m\times 1\) vector of exogenous
structural shocks to the model. \(E_tX_{t+1}\) denotes the
expecation of the \(t+1\) endogenous variables based on the
information available to decision makers in the model as of time period
\(t\).

The function \(f\) is often nonlinear. Because the values of the
endogenous variables in period \(t\) depend on the expected future
values of those variables, it is not in general possible to compute the
equilibirum of the model by working directly with the function
\(f\). Instead it is often convenient to work with a log-linear
approximation to the equilibrium conditions around a non-stochastic
steady state. In many cases, the log-linear approximation can be written
in the following form:
\phantomsection\label{\detokenize{examples:equation-examples:1}}\label{equation:examples:examples:1}\begin{align}
A E_t\left[ x_{t+1} \right] & = B x_t + \left[ \begin{array}{c} \epsilon_{t+1} \\ 0 \end{array} \right],
\end{align}
where the vector \(x_{t}\) denotes the log deviation of the
variables in \(X_t\) from their steady state values. The variables
in \(x_t\) are grouped in a specific way: \(x_t = [s_t; u_t]\)
where \(s_t\) is an \(n_s \times 1\) vector of predetermined
(state) variables and \(u_t\) is an \(n_u \times 1\) vector of
nonpredetermined (forward-looking) variables. \(\epsilon_{t+1}\) is
an \(n_s\times 1\) vector of i.i.d. shocks to the state variables
\(s_{t+1}\). \(\epsilon_{t+1}\) has mean 0 and diagonal
covariance matrix \(\Sigma\). The solution to the model is a pair of
matrices \(F\) and \(P\) such that:
\phantomsection\label{\detokenize{examples:equation-examples:2}}\label{equation:examples:examples:2}\begin{align}
u_t  &  = Fs_t\\
s_{t+1} & = Ps_t + \epsilon_{t+1}.
\end{align}
The matrices \(F\) and \(P\) are obtained using the \href{http://www.sciencedirect.com/science/article/pii/S0165188999000457}{Klein
(2000)}
solution method which is based on the generalized Schur factorization of
the marices \(A\) and \(B\). The solution routine incorporates
many aspects of his program for Matlab
\sphinxcode{{}`solab.m} \textless{}\url{http://paulklein.ca/newsite/codes/codes.php}\textgreater{}{}`\_\_.

This package defines a \sphinxcode{linearsolve.model} class. An instance of the
\sphinxcode{linearsolve.model} has the following methods:
\begin{enumerate}
\item {} 
\sphinxcode{compute\_ss(guess,method,options)}: Computes the steady state of
the nonlinear model.

\item {} 
\sphinxcode{set\_ss(steady\_state)}: Sets the steady state \sphinxcode{.ss} attribute of
the instance.

\item {} 
\sphinxcode{log\_linear\_approximation(steady\_state,isloglinear)}:
Log-linearizes the nonlinear model and constructs the matrices
\(A\) and \(B\).

\item {} 
\sphinxcode{klein(a,b)}: Solves the linear model using Klein's solution
method.

\item {} 
\sphinxcode{approximate\_and\_solve(isloglinear)}: Approximates and solves the
model by combining the previous two methods.

\item {} 
\sphinxcode{impulse(T,t0,shock,percent)}: Computes impulse responses for
shocks to each endogenous state variable.

\item {} 
\sphinxcode{approximated(round,precision)}: Returns a string containing the
log-linear approximation to the equilibrium conditions of the model.

\item {} 
\sphinxcode{solved(round,precision)}: Returns a string containing the solution
to the log-linear approximation of the model.

\end{enumerate}

In this notebook, I demonstrate how to use the module to simulate two
basic business cycle models: an real business cycle (RBC) model and a
new-Keynesian business cycle model.


\subsection{Example 1: A quick example.}
\label{\detokenize{examples:Example-1:-A-quick-example.}}
Here I demonstrate how how relatively straightforward it is to
appoximate, solve, and simulate a DSGE model using \sphinxcode{linearsolve}. In
the example that follows, I describe the procedure more carefully.
\phantomsection\label{\detokenize{examples:equation-examples:3}}\label{equation:examples:examples:3}\begin{align}
C_t^{-\sigma} & = \beta E_t \left[C_{t+1}^{-\sigma}(\alpha A_{t+1} K_{t+1}^{\alpha-1} + 1 - \delta)\right]\\
C_t + K_{t+1} & = A_t K_t^{\alpha} + (1-\delta)K_t\\
\log A_{t+1} & = \rho_a \log A_{t} + \epsilon_{t+1}
\end{align}
In the block of code that immediately follows, I input the model, solve
for the steady state, compute the log-linear approximation of the
equilibirum conditions, and compute some impulse responses following a
shock to technology \(A_t\).

\begin{Verbatim}[commandchars=\\\{\}]
\textcolor{nbsphinxin}{In [2]: }\PYG{c+c1}{\PYGZsh{} Input model parameters}
        \PYG{n}{parameters} \PYG{o}{=} \PYG{n}{pd}\PYG{o}{.}\PYG{n}{Series}\PYG{p}{(}\PYG{p}{)}
        \PYG{n}{parameters}\PYG{p}{[}\PYG{l+s+s1}{\PYGZsq{}}\PYG{l+s+s1}{alpha}\PYG{l+s+s1}{\PYGZsq{}}\PYG{p}{]}  \PYG{o}{=} \PYG{o}{.}\PYG{l+m+mi}{35}
        \PYG{n}{parameters}\PYG{p}{[}\PYG{l+s+s1}{\PYGZsq{}}\PYG{l+s+s1}{beta}\PYG{l+s+s1}{\PYGZsq{}}\PYG{p}{]}  \PYG{o}{=} \PYG{l+m+mf}{0.99}
        \PYG{n}{parameters}\PYG{p}{[}\PYG{l+s+s1}{\PYGZsq{}}\PYG{l+s+s1}{delta}\PYG{l+s+s1}{\PYGZsq{}}\PYG{p}{]}   \PYG{o}{=} \PYG{l+m+mf}{0.025}
        \PYG{n}{parameters}\PYG{p}{[}\PYG{l+s+s1}{\PYGZsq{}}\PYG{l+s+s1}{rhoa}\PYG{l+s+s1}{\PYGZsq{}}\PYG{p}{]} \PYG{o}{=} \PYG{o}{.}\PYG{l+m+mi}{9}
        \PYG{n}{parameters}\PYG{p}{[}\PYG{l+s+s1}{\PYGZsq{}}\PYG{l+s+s1}{sigma}\PYG{l+s+s1}{\PYGZsq{}}\PYG{p}{]} \PYG{o}{=} \PYG{l+m+mf}{1.5}
        \PYG{n}{parameters}\PYG{p}{[}\PYG{l+s+s1}{\PYGZsq{}}\PYG{l+s+s1}{A}\PYG{l+s+s1}{\PYGZsq{}}\PYG{p}{]} \PYG{o}{=} \PYG{l+m+mi}{1}
        
        \PYG{c+c1}{\PYGZsh{} Funtion that evaluates the equilibrium conditions}
        \PYG{k}{def} \PYG{n+nf}{equilibrium\PYGZus{}equations}\PYG{p}{(}\PYG{n}{variables\PYGZus{}forward}\PYG{p}{,}\PYG{n}{variables\PYGZus{}current}\PYG{p}{,}\PYG{n}{parameters}\PYG{p}{)}\PYG{p}{:}
        
            \PYG{c+c1}{\PYGZsh{} Parameters}
            \PYG{n}{p} \PYG{o}{=} \PYG{n}{parameters}
        
            \PYG{c+c1}{\PYGZsh{} Variables}
            \PYG{n}{fwd} \PYG{o}{=} \PYG{n}{variables\PYGZus{}forward}
            \PYG{n}{cur} \PYG{o}{=} \PYG{n}{variables\PYGZus{}current}
        
            \PYG{c+c1}{\PYGZsh{} Household Euler equation}
            \PYG{n}{euler\PYGZus{}eqn} \PYG{o}{=} \PYG{n}{p}\PYG{o}{.}\PYG{n}{beta}\PYG{o}{*}\PYG{n}{fwd}\PYG{o}{.}\PYG{n}{c}\PYG{o}{*}\PYG{o}{*}\PYG{o}{\PYGZhy{}}\PYG{n}{p}\PYG{o}{.}\PYG{n}{sigma}\PYG{o}{*}\PYG{p}{(}\PYG{n}{p}\PYG{o}{.}\PYG{n}{alpha}\PYG{o}{*}\PYG{n}{cur}\PYG{o}{.}\PYG{n}{a}\PYG{o}{*}\PYG{n}{fwd}\PYG{o}{.}\PYG{n}{k}\PYG{o}{*}\PYG{o}{*}\PYG{p}{(}\PYG{n}{p}\PYG{o}{.}\PYG{n}{alpha}\PYG{o}{\PYGZhy{}}\PYG{l+m+mi}{1}\PYG{p}{)}\PYG{o}{+}\PYG{l+m+mi}{1}\PYG{o}{\PYGZhy{}}\PYG{n}{p}\PYG{o}{.}\PYG{n}{delta}\PYG{p}{)} \PYG{o}{\PYGZhy{}} \PYG{n}{cur}\PYG{o}{.}\PYG{n}{c}\PYG{o}{*}\PYG{o}{*}\PYG{o}{\PYGZhy{}}\PYG{n}{p}\PYG{o}{.}\PYG{n}{sigma}
        
            \PYG{c+c1}{\PYGZsh{} Goods market clearing}
            \PYG{n}{market\PYGZus{}clearing} \PYG{o}{=} \PYG{n}{cur}\PYG{o}{.}\PYG{n}{c} \PYG{o}{+} \PYG{n}{fwd}\PYG{o}{.}\PYG{n}{k} \PYG{o}{\PYGZhy{}} \PYG{p}{(}\PYG{l+m+mi}{1}\PYG{o}{\PYGZhy{}}\PYG{n}{p}\PYG{o}{.}\PYG{n}{delta}\PYG{p}{)}\PYG{o}{*}\PYG{n}{cur}\PYG{o}{.}\PYG{n}{k} \PYG{o}{\PYGZhy{}} \PYG{n}{cur}\PYG{o}{.}\PYG{n}{a}\PYG{o}{*}\PYG{n}{cur}\PYG{o}{.}\PYG{n}{k}\PYG{o}{*}\PYG{o}{*}\PYG{n}{p}\PYG{o}{.}\PYG{n}{alpha}
        
            \PYG{c+c1}{\PYGZsh{} Exogenous technology}
        \PYG{c+c1}{\PYGZsh{}     technology\PYGZus{}proc = cur.a**p.rhoa \PYGZhy{} fwd.a}
            \PYG{n}{technology\PYGZus{}proc} \PYG{o}{=} \PYG{n}{p}\PYG{o}{.}\PYG{n}{rhoa}\PYG{o}{*}\PYG{n}{np}\PYG{o}{.}\PYG{n}{log}\PYG{p}{(}\PYG{n}{cur}\PYG{o}{.}\PYG{n}{a}\PYG{p}{)} \PYG{o}{\PYGZhy{}} \PYG{n}{np}\PYG{o}{.}\PYG{n}{log}\PYG{p}{(}\PYG{n}{fwd}\PYG{o}{.}\PYG{n}{a}\PYG{p}{)}
        
            \PYG{c+c1}{\PYGZsh{} Stack equilibrium conditions into a numpy array}
            \PYG{k}{return} \PYG{n}{np}\PYG{o}{.}\PYG{n}{array}\PYG{p}{(}\PYG{p}{[}
                    \PYG{n}{euler\PYGZus{}eqn}\PYG{p}{,}
                    \PYG{n}{market\PYGZus{}clearing}\PYG{p}{,}
                    \PYG{n}{technology\PYGZus{}proc}
                \PYG{p}{]}\PYG{p}{)}
        
        \PYG{c+c1}{\PYGZsh{} Initialize the model}
        \PYG{n}{model} \PYG{o}{=} \PYG{n}{ls}\PYG{o}{.}\PYG{n}{model}\PYG{p}{(}\PYG{n}{equations} \PYG{o}{=} \PYG{n}{equilibrium\PYGZus{}equations}\PYG{p}{,}
                         \PYG{n}{nstates}\PYG{o}{=}\PYG{l+m+mi}{2}\PYG{p}{,}
                         \PYG{n}{varNames}\PYG{o}{=}\PYG{p}{[}\PYG{l+s+s1}{\PYGZsq{}}\PYG{l+s+s1}{a}\PYG{l+s+s1}{\PYGZsq{}}\PYG{p}{,}\PYG{l+s+s1}{\PYGZsq{}}\PYG{l+s+s1}{k}\PYG{l+s+s1}{\PYGZsq{}}\PYG{p}{,}\PYG{l+s+s1}{\PYGZsq{}}\PYG{l+s+s1}{c}\PYG{l+s+s1}{\PYGZsq{}}\PYG{p}{]}\PYG{p}{,}
                         \PYG{n}{shockNames}\PYG{o}{=}\PYG{p}{[}\PYG{l+s+s1}{\PYGZsq{}}\PYG{l+s+s1}{eA}\PYG{l+s+s1}{\PYGZsq{}}\PYG{p}{,}\PYG{l+s+s1}{\PYGZsq{}}\PYG{l+s+s1}{eK}\PYG{l+s+s1}{\PYGZsq{}}\PYG{p}{]}\PYG{p}{,}
                         \PYG{n}{parameters} \PYG{o}{=} \PYG{n}{parameters}\PYG{p}{)}
        
        \PYG{c+c1}{\PYGZsh{} Compute the steady state numerically}
        \PYG{n}{guess} \PYG{o}{=} \PYG{p}{[}\PYG{l+m+mi}{1}\PYG{p}{,}\PYG{l+m+mi}{1}\PYG{p}{,}\PYG{l+m+mi}{1}\PYG{p}{]}
        \PYG{n}{model}\PYG{o}{.}\PYG{n}{compute\PYGZus{}ss}\PYG{p}{(}\PYG{n}{guess}\PYG{p}{)}
        
        \PYG{c+c1}{\PYGZsh{} model.ss}
        
        \PYG{c+c1}{\PYGZsh{} Find the log\PYGZhy{}linear approximation around the non\PYGZhy{}stochastic steady state and solve}
        \PYG{n}{model}\PYG{o}{.}\PYG{n}{approximate\PYGZus{}and\PYGZus{}solve}\PYG{p}{(}\PYG{p}{)}
        
        \PYG{c+c1}{\PYGZsh{} Compute impulse responses and plot}
        \PYG{n}{model}\PYG{o}{.}\PYG{n}{impulse}\PYG{p}{(}\PYG{n}{T}\PYG{o}{=}\PYG{l+m+mi}{41}\PYG{p}{,}\PYG{n}{t0}\PYG{o}{=}\PYG{l+m+mi}{5}\PYG{p}{,}\PYG{n}{shock}\PYG{o}{=}\PYG{n+nb+bp}{None}\PYG{p}{)}
        
        \PYG{n}{fig} \PYG{o}{=} \PYG{n}{plt}\PYG{o}{.}\PYG{n}{figure}\PYG{p}{(}\PYG{n}{figsize}\PYG{o}{=}\PYG{p}{(}\PYG{l+m+mi}{12}\PYG{p}{,}\PYG{l+m+mi}{4}\PYG{p}{)}\PYG{p}{)}
        \PYG{n}{ax1} \PYG{o}{=}\PYG{n}{fig}\PYG{o}{.}\PYG{n}{add\PYGZus{}subplot}\PYG{p}{(}\PYG{l+m+mi}{1}\PYG{p}{,}\PYG{l+m+mi}{2}\PYG{p}{,}\PYG{l+m+mi}{1}\PYG{p}{)}
        \PYG{n}{model}\PYG{o}{.}\PYG{n}{irs}\PYG{p}{[}\PYG{l+s+s1}{\PYGZsq{}}\PYG{l+s+s1}{eA}\PYG{l+s+s1}{\PYGZsq{}}\PYG{p}{]}\PYG{p}{[}\PYG{p}{[}\PYG{l+s+s1}{\PYGZsq{}}\PYG{l+s+s1}{a}\PYG{l+s+s1}{\PYGZsq{}}\PYG{p}{,}\PYG{l+s+s1}{\PYGZsq{}}\PYG{l+s+s1}{k}\PYG{l+s+s1}{\PYGZsq{}}\PYG{p}{,}\PYG{l+s+s1}{\PYGZsq{}}\PYG{l+s+s1}{c}\PYG{l+s+s1}{\PYGZsq{}}\PYG{p}{]}\PYG{p}{]}\PYG{o}{.}\PYG{n}{plot}\PYG{p}{(}\PYG{n}{lw}\PYG{o}{=}\PYG{l+s+s1}{\PYGZsq{}}\PYG{l+s+s1}{5}\PYG{l+s+s1}{\PYGZsq{}}\PYG{p}{,}\PYG{n}{alpha}\PYG{o}{=}\PYG{l+m+mf}{0.5}\PYG{p}{,}\PYG{n}{grid}\PYG{o}{=}\PYG{n+nb+bp}{True}\PYG{p}{,}\PYG{n}{ax}\PYG{o}{=}\PYG{n}{ax1}\PYG{p}{)}\PYG{o}{.}\PYG{n}{legend}\PYG{p}{(}\PYG{n}{loc}\PYG{o}{=}\PYG{l+s+s1}{\PYGZsq{}}\PYG{l+s+s1}{upper right}\PYG{l+s+s1}{\PYGZsq{}}\PYG{p}{,}\PYG{n}{ncol}\PYG{o}{=}\PYG{l+m+mi}{3}\PYG{p}{)}
        \PYG{n}{ax2} \PYG{o}{=}\PYG{n}{fig}\PYG{o}{.}\PYG{n}{add\PYGZus{}subplot}\PYG{p}{(}\PYG{l+m+mi}{1}\PYG{p}{,}\PYG{l+m+mi}{2}\PYG{p}{,}\PYG{l+m+mi}{2}\PYG{p}{)}
        \PYG{n}{model}\PYG{o}{.}\PYG{n}{irs}\PYG{p}{[}\PYG{l+s+s1}{\PYGZsq{}}\PYG{l+s+s1}{eA}\PYG{l+s+s1}{\PYGZsq{}}\PYG{p}{]}\PYG{p}{[}\PYG{p}{[}\PYG{l+s+s1}{\PYGZsq{}}\PYG{l+s+s1}{eA}\PYG{l+s+s1}{\PYGZsq{}}\PYG{p}{,}\PYG{l+s+s1}{\PYGZsq{}}\PYG{l+s+s1}{a}\PYG{l+s+s1}{\PYGZsq{}}\PYG{p}{]}\PYG{p}{]}\PYG{o}{.}\PYG{n}{plot}\PYG{p}{(}\PYG{n}{lw}\PYG{o}{=}\PYG{l+s+s1}{\PYGZsq{}}\PYG{l+s+s1}{5}\PYG{l+s+s1}{\PYGZsq{}}\PYG{p}{,}\PYG{n}{alpha}\PYG{o}{=}\PYG{l+m+mf}{0.5}\PYG{p}{,}\PYG{n}{grid}\PYG{o}{=}\PYG{n+nb+bp}{True}\PYG{p}{,}\PYG{n}{ax}\PYG{o}{=}\PYG{n}{ax2}\PYG{p}{)}\PYG{o}{.}\PYG{n}{legend}\PYG{p}{(}\PYG{n}{loc}\PYG{o}{=}\PYG{l+s+s1}{\PYGZsq{}}\PYG{l+s+s1}{upper right}\PYG{l+s+s1}{\PYGZsq{}}\PYG{p}{,}\PYG{n}{ncol}\PYG{o}{=}\PYG{l+m+mi}{2}\PYG{p}{)}
\end{Verbatim}

\begin{Verbatim}[commandchars=\\\{\}]
\textcolor{nbsphinxout}{Out[2]: }\PYGZlt{}matplotlib.legend.Legend at 0x116dc6390\PYGZgt{}
\end{Verbatim}

\noindent\sphinxincludegraphics{{examples_3_1}.png}


\subsection{Example 2: A slightly more elaborate model with explanation}
\label{\detokenize{examples:Example-2:-A-slightly-more-elaborate-model-with-explanation}}
Consider the equilibrium conditions for a basic RBC model without labor:
\phantomsection\label{\detokenize{examples:equation-examples:4}}\label{equation:examples:examples:4}\begin{align}
C_t^{-\sigma} & = \beta E_t \left[C_{t+1}^{-\sigma}(\alpha A_{t+1} K_{t+1}^{\alpha-1} + 1 - \delta)\right]\\
Y_t & = A_t K_t^{\alpha}\\
I_t & = K_{t+1} - (1-\delta)K_t\\
Y_t & = C_t + I_t\\
\log A_t & = \rho_a \log A_{t-1} + \epsilon_t
\end{align}
In the nonstochastic steady state, we have:
\phantomsection\label{\detokenize{examples:equation-examples:5}}\label{equation:examples:examples:5}\begin{align}
K & = \left(\frac{\alpha A}{1/\beta+\delta-1}\right)^{\frac{1}{1-\alpha}}\\
Y & = AK^{\alpha}\\
I & = \delta K\\
C & = Y - I
\end{align}
Given values for the parameters \(\beta\), \(\sigma\),
\(\alpha\), \(\delta\), and \(A\), steady state values of
capital, output, investment, and consumption are easily computed.


\subsubsection{Initializing the model in \sphinxstyleliteralintitle{linearsolve}}
\label{\detokenize{examples:Initializing-the-model-in-linearsolve}}
To initialize the model, we need to first set the model's parameters. We
do this by creating a Pandas Series variable called \sphinxcode{parameters}:

\begin{Verbatim}[commandchars=\\\{\}]
\textcolor{nbsphinxin}{In [3]: }\PYG{c+c1}{\PYGZsh{} Input model parameters}
        \PYG{n}{parameters} \PYG{o}{=} \PYG{n}{pd}\PYG{o}{.}\PYG{n}{Series}\PYG{p}{(}\PYG{p}{)}
        \PYG{n}{parameters}\PYG{p}{[}\PYG{l+s+s1}{\PYGZsq{}}\PYG{l+s+s1}{alpha}\PYG{l+s+s1}{\PYGZsq{}}\PYG{p}{]}  \PYG{o}{=} \PYG{o}{.}\PYG{l+m+mi}{35}
        \PYG{n}{parameters}\PYG{p}{[}\PYG{l+s+s1}{\PYGZsq{}}\PYG{l+s+s1}{beta}\PYG{l+s+s1}{\PYGZsq{}}\PYG{p}{]}  \PYG{o}{=} \PYG{l+m+mf}{0.99}
        \PYG{n}{parameters}\PYG{p}{[}\PYG{l+s+s1}{\PYGZsq{}}\PYG{l+s+s1}{delta}\PYG{l+s+s1}{\PYGZsq{}}\PYG{p}{]}   \PYG{o}{=} \PYG{l+m+mf}{0.025}
        \PYG{n}{parameters}\PYG{p}{[}\PYG{l+s+s1}{\PYGZsq{}}\PYG{l+s+s1}{rhoa}\PYG{l+s+s1}{\PYGZsq{}}\PYG{p}{]} \PYG{o}{=} \PYG{o}{.}\PYG{l+m+mi}{9}
        \PYG{n}{parameters}\PYG{p}{[}\PYG{l+s+s1}{\PYGZsq{}}\PYG{l+s+s1}{sigma}\PYG{l+s+s1}{\PYGZsq{}}\PYG{p}{]} \PYG{o}{=} \PYG{l+m+mf}{1.5}
        \PYG{n}{parameters}\PYG{p}{[}\PYG{l+s+s1}{\PYGZsq{}}\PYG{l+s+s1}{A}\PYG{l+s+s1}{\PYGZsq{}}\PYG{p}{]} \PYG{o}{=} \PYG{l+m+mi}{1}
\end{Verbatim}

Next, we need to define a function that returns the equilibrium
conditions of the model. The function will take as inputs two vectors:
one vector of ``current'' variables and another of ``forward-looking'' or
one-period-ahead variables. The function will return an array that
represents the equilibirum conditions of the model. We'll enter each
equation with all variables moved to one side of the equals sign. For
example, here's how we'll enter the produciton fucntion:

\sphinxcode{production\_function = technology\_current*capital\_current**alpha - output\_curent}

Here the variable \sphinxcode{production\_function} stores the production function
equation set equal to zero. We can enter the equations in almost any way
we want. For example, we could also have entered the production function
this way:

\sphinxcode{production\_function = 1 - output\_curent/technology\_current/capital\_current**alpha}

One more thing to consider: the natural log in the equation describing
the evolution of total factor productivity will create problems for the
solution routine later on. So rewrite the equation as:
\phantomsection\label{\detokenize{examples:equation-examples:6}}\label{equation:examples:examples:6}\begin{align}
A_{t+1} & =  A_{t}^{\rho_a}e^{\epsilon_{t+1}}\\
\end{align}
So the complete system of equations that we enter into the program looks
like:
\phantomsection\label{\detokenize{examples:equation-examples:7}}\label{equation:examples:examples:7}\begin{align}
C_t^{-\sigma} & = \beta E_t \left[C_{t+1}^{-\sigma}(\alpha Y_{t+1} /K_{t+1}+ 1 - \delta)\right]\\
Y_t & = A_t K_t^{\alpha}\\
I_t & = K_{t+1} - (1-\delta)K_t\\
Y_t & = C_t + I_t\\
A_{t+1} & =  A_{t}^{\rho_a}e^{\epsilon_{t+1}}
\end{align}
Now let's define the function that returns the equilibrium conditions:

\begin{Verbatim}[commandchars=\\\{\}]
\textcolor{nbsphinxin}{In [4]: }\PYG{k}{def} \PYG{n+nf}{equilibrium\PYGZus{}equations}\PYG{p}{(}\PYG{n}{variables\PYGZus{}forward}\PYG{p}{,}\PYG{n}{variables\PYGZus{}current}\PYG{p}{,}\PYG{n}{parameters}\PYG{p}{)}\PYG{p}{:}
        
            \PYG{c+c1}{\PYGZsh{} Parameters}
            \PYG{n}{p} \PYG{o}{=} \PYG{n}{parameters}
        
            \PYG{c+c1}{\PYGZsh{} Variables}
            \PYG{n}{fwd} \PYG{o}{=} \PYG{n}{variables\PYGZus{}forward}
            \PYG{n}{cur} \PYG{o}{=} \PYG{n}{variables\PYGZus{}current}
        
            \PYG{c+c1}{\PYGZsh{} Household Euler equation}
            \PYG{n}{euler\PYGZus{}eqn} \PYG{o}{=} \PYG{n}{p}\PYG{o}{.}\PYG{n}{beta}\PYG{o}{*}\PYG{n}{fwd}\PYG{o}{.}\PYG{n}{c}\PYG{o}{*}\PYG{o}{*}\PYG{o}{\PYGZhy{}}\PYG{n}{p}\PYG{o}{.}\PYG{n}{sigma}\PYG{o}{*}\PYG{p}{(}\PYG{n}{p}\PYG{o}{.}\PYG{n}{alpha}\PYG{o}{*}\PYG{n}{fwd}\PYG{o}{.}\PYG{n}{y}\PYG{o}{/}\PYG{n}{fwd}\PYG{o}{.}\PYG{n}{k}\PYG{o}{+}\PYG{l+m+mi}{1}\PYG{o}{\PYGZhy{}}\PYG{n}{p}\PYG{o}{.}\PYG{n}{delta}\PYG{p}{)} \PYG{o}{\PYGZhy{}} \PYG{n}{cur}\PYG{o}{.}\PYG{n}{c}\PYG{o}{*}\PYG{o}{*}\PYG{o}{\PYGZhy{}}\PYG{n}{p}\PYG{o}{.}\PYG{n}{sigma}
        
            \PYG{c+c1}{\PYGZsh{} Production function}
            \PYG{n}{production\PYGZus{}fuction} \PYG{o}{=}  \PYG{n}{cur}\PYG{o}{.}\PYG{n}{a}\PYG{o}{*}\PYG{n}{cur}\PYG{o}{.}\PYG{n}{k}\PYG{o}{*}\PYG{o}{*}\PYG{n}{p}\PYG{o}{.}\PYG{n}{alpha} \PYG{o}{\PYGZhy{}} \PYG{n}{cur}\PYG{o}{.}\PYG{n}{y}
        
            \PYG{c+c1}{\PYGZsh{} Capital evolution}
            \PYG{n}{capital\PYGZus{}evolution} \PYG{o}{=} \PYG{n}{fwd}\PYG{o}{.}\PYG{n}{k} \PYG{o}{\PYGZhy{}} \PYG{p}{(}\PYG{l+m+mi}{1}\PYG{o}{\PYGZhy{}}\PYG{n}{p}\PYG{o}{.}\PYG{n}{delta}\PYG{p}{)}\PYG{o}{*}\PYG{n}{cur}\PYG{o}{.}\PYG{n}{k} \PYG{o}{\PYGZhy{}} \PYG{n}{cur}\PYG{o}{.}\PYG{n}{i}
        
            \PYG{c+c1}{\PYGZsh{} Goods market clearing}
            \PYG{n}{market\PYGZus{}clearing} \PYG{o}{=} \PYG{n}{cur}\PYG{o}{.}\PYG{n}{c} \PYG{o}{+} \PYG{n}{cur}\PYG{o}{.}\PYG{n}{i} \PYG{o}{\PYGZhy{}} \PYG{n}{cur}\PYG{o}{.}\PYG{n}{y}
        
            \PYG{c+c1}{\PYGZsh{} Exogenous technology}
            \PYG{n}{technology\PYGZus{}proc} \PYG{o}{=} \PYG{n}{cur}\PYG{o}{.}\PYG{n}{a}\PYG{o}{*}\PYG{o}{*}\PYG{n}{p}\PYG{o}{.}\PYG{n}{rhoa}\PYG{o}{\PYGZhy{}} \PYG{n}{fwd}\PYG{o}{.}\PYG{n}{a}
        
            \PYG{c+c1}{\PYGZsh{} Stack equilibrium conditions into a numpy array}
            \PYG{k}{return} \PYG{n}{np}\PYG{o}{.}\PYG{n}{array}\PYG{p}{(}\PYG{p}{[}
                    \PYG{n}{euler\PYGZus{}eqn}\PYG{p}{,}
                    \PYG{n}{production\PYGZus{}fuction}\PYG{p}{,}
                    \PYG{n}{capital\PYGZus{}evolution}\PYG{p}{,}
                    \PYG{n}{market\PYGZus{}clearing}\PYG{p}{,}
                    \PYG{n}{technology\PYGZus{}proc}
                \PYG{p}{]}\PYG{p}{)}
\end{Verbatim}

Notice that inside the function we have to define the variables of the
model form the elements of the input vectors \sphinxcode{variables\_forward} and
\sphinxcode{variables\_current}. It is \sphinxstyleemphasis{essential} that the predetermined or state
variables are ordered first.


\subsubsection{Initializing the model}
\label{\detokenize{examples:Initializing-the-model}}
To initialize the model, we need to specify the number of state
variables in the model, the names of the endogenous varaibles in the
same order used in the \sphinxcode{equilibrium\_equations} function, and the names
of the exogenous shocks to the model.

\begin{Verbatim}[commandchars=\\\{\}]
\textcolor{nbsphinxin}{In [5]: }\PYG{c+c1}{\PYGZsh{} Initialize the model}
        \PYG{n}{rbc} \PYG{o}{=} \PYG{n}{ls}\PYG{o}{.}\PYG{n}{model}\PYG{p}{(}\PYG{n}{equations} \PYG{o}{=} \PYG{n}{equilibrium\PYGZus{}equations}\PYG{p}{,}
                       \PYG{n}{nstates}\PYG{o}{=}\PYG{l+m+mi}{2}\PYG{p}{,}
                       \PYG{n}{varNames}\PYG{o}{=}\PYG{p}{[}\PYG{l+s+s1}{\PYGZsq{}}\PYG{l+s+s1}{a}\PYG{l+s+s1}{\PYGZsq{}}\PYG{p}{,}\PYG{l+s+s1}{\PYGZsq{}}\PYG{l+s+s1}{k}\PYG{l+s+s1}{\PYGZsq{}}\PYG{p}{,}\PYG{l+s+s1}{\PYGZsq{}}\PYG{l+s+s1}{c}\PYG{l+s+s1}{\PYGZsq{}}\PYG{p}{,}\PYG{l+s+s1}{\PYGZsq{}}\PYG{l+s+s1}{y}\PYG{l+s+s1}{\PYGZsq{}}\PYG{p}{,}\PYG{l+s+s1}{\PYGZsq{}}\PYG{l+s+s1}{i}\PYG{l+s+s1}{\PYGZsq{}}\PYG{p}{]}\PYG{p}{,}
                       \PYG{n}{shockNames}\PYG{o}{=}\PYG{p}{[}\PYG{l+s+s1}{\PYGZsq{}}\PYG{l+s+s1}{eA}\PYG{l+s+s1}{\PYGZsq{}}\PYG{p}{,}\PYG{l+s+s1}{\PYGZsq{}}\PYG{l+s+s1}{eK}\PYG{l+s+s1}{\PYGZsq{}}\PYG{p}{]}\PYG{p}{,}
                       \PYG{n}{parameters}\PYG{o}{=}\PYG{n}{parameters}\PYG{p}{)}
\end{Verbatim}

The solution routine solves the model as if there were a separate
exogenous shock for each state variable and that's why I initialized the
model with two exogenous shocks \sphinxcode{eA} and \sphinxcode{eK} even though the RBC
model only has one exogenous shock.


\subsubsection{Steady state}
\label{\detokenize{examples:Steady-state}}
Next, we need to compute the nonstochastic steady state of the model.
The \sphinxcode{.compute\_ss} method can be used to compute the steady state
numerically. The method's default is to use scipy's \sphinxcode{fsolve} function,
but other scipy root-finding functions can be used: \sphinxcode{root},
\sphinxcode{broyden1}, and \sphinxcode{broyden2}. The optional argument \sphinxcode{options} lets
the user pass keywords directly to the optimization function. Check out
the documentation for Scipy's nonlinear solvers here:
\url{http://docs.scipy.org/doc/scipy/reference/optimize.html}

\begin{Verbatim}[commandchars=\\\{\}]
\textcolor{nbsphinxin}{In [6]: }\PYG{c+c1}{\PYGZsh{} Compute the steady state numerically}
        \PYG{n}{guess} \PYG{o}{=} \PYG{p}{[}\PYG{l+m+mi}{1}\PYG{p}{,}\PYG{l+m+mi}{1}\PYG{p}{,}\PYG{l+m+mi}{1}\PYG{p}{,}\PYG{l+m+mi}{1}\PYG{p}{,}\PYG{l+m+mi}{1}\PYG{p}{]}
        \PYG{n}{rbc}\PYG{o}{.}\PYG{n}{compute\PYGZus{}ss}\PYG{p}{(}\PYG{n}{guess}\PYG{p}{)}
        \PYG{k}{print}\PYG{p}{(}\PYG{n}{rbc}\PYG{o}{.}\PYG{n}{ss}\PYG{p}{)}
\end{Verbatim}
% This comment is needed to force a line break for adjacent ANSI cells
\begin{OriginalVerbatim}[commandchars=\\\{\}]
a     1.000000
k    34.398226
c     2.589794
y     3.449750
i     0.859956
dtype: float64
\end{OriginalVerbatim}
Note that the steady state is returned as a Pandas Series.
Alternatively, you could compute the steady state directly and then sent
the \sphinxcode{rbc.ss} attribute:

\begin{Verbatim}[commandchars=\\\{\}]
\textcolor{nbsphinxin}{In [7]: }\PYG{c+c1}{\PYGZsh{} Steady state solution}
        \PYG{n}{p} \PYG{o}{=} \PYG{n}{parameters}
        \PYG{n}{K} \PYG{o}{=} \PYG{p}{(}\PYG{n}{p}\PYG{o}{.}\PYG{n}{alpha}\PYG{o}{*}\PYG{n}{p}\PYG{o}{.}\PYG{n}{A}\PYG{o}{/}\PYG{p}{(}\PYG{l+m+mi}{1}\PYG{o}{/}\PYG{n}{p}\PYG{o}{.}\PYG{n}{beta}\PYG{o}{+}\PYG{n}{p}\PYG{o}{.}\PYG{n}{delta}\PYG{o}{\PYGZhy{}}\PYG{l+m+mi}{1}\PYG{p}{)}\PYG{p}{)}\PYG{o}{*}\PYG{o}{*}\PYG{p}{(}\PYG{l+m+mi}{1}\PYG{o}{/}\PYG{p}{(}\PYG{l+m+mi}{1}\PYG{o}{\PYGZhy{}}\PYG{n}{p}\PYG{o}{.}\PYG{n}{alpha}\PYG{p}{)}\PYG{p}{)}
        \PYG{n}{C} \PYG{o}{=} \PYG{n}{p}\PYG{o}{.}\PYG{n}{A}\PYG{o}{*}\PYG{n}{K}\PYG{o}{*}\PYG{o}{*}\PYG{n}{p}\PYG{o}{.}\PYG{n}{alpha} \PYG{o}{\PYGZhy{}} \PYG{n}{p}\PYG{o}{.}\PYG{n}{delta}\PYG{o}{*}\PYG{n}{K}
        \PYG{n}{Y} \PYG{o}{=} \PYG{n}{p}\PYG{o}{.}\PYG{n}{A}\PYG{o}{*}\PYG{n}{K}\PYG{o}{*}\PYG{o}{*}\PYG{n}{p}\PYG{o}{.}\PYG{n}{alpha}
        \PYG{n}{I} \PYG{o}{=} \PYG{n}{Y} \PYG{o}{\PYGZhy{}} \PYG{n}{C}
        
        \PYG{n}{rbc}\PYG{o}{.}\PYG{n}{set\PYGZus{}ss}\PYG{p}{(}\PYG{p}{[}\PYG{n}{p}\PYG{o}{.}\PYG{n}{A}\PYG{p}{,}\PYG{n}{K}\PYG{p}{,}\PYG{n}{C}\PYG{p}{,}\PYG{n}{Y}\PYG{p}{,}\PYG{n}{I}\PYG{p}{]}\PYG{p}{)}
        \PYG{k}{print}\PYG{p}{(}\PYG{n}{rbc}\PYG{o}{.}\PYG{n}{ss}\PYG{p}{)}
\end{Verbatim}
% This comment is needed to force a line break for adjacent ANSI cells
\begin{OriginalVerbatim}[commandchars=\\\{\}]
[  1.          34.39822605   2.58979429   3.44974994   0.85995565]
\end{OriginalVerbatim}

\subsubsection{Log-linearization and solution}
\label{\detokenize{examples:Log-linearization-and-solution}}
Now we use the \sphinxcode{.log\_linear} method to find the log-linear
appxoximation to the model's equilibrium conditions. That is, we'll
transform the nonlinear model into a linear model in which all variables
are expressed as log-deviations from the steady state. Specifically,
we'll compute the matrices \(A\) and \(B\) that satisfy:
\phantomsection\label{\detokenize{examples:equation-examples:8}}\label{equation:examples:examples:8}\begin{align}
A E_t\left[ x_{t+1} \right] & = B x_t + \left[ \begin{array}{c} \epsilon_{t+1} \\ 0 \end{array} \right],
\end{align}
where the vector \(x_{t}\) denotes the log deviation of the
endogenous variables from their steady state values.

\begin{Verbatim}[commandchars=\\\{\}]
\textcolor{nbsphinxin}{In [8]: }\PYG{c+c1}{\PYGZsh{} Find the log\PYGZhy{}linear approximation around the non\PYGZhy{}stochastic steady state}
        \PYG{n}{rbc}\PYG{o}{.}\PYG{n}{log\PYGZus{}linear\PYGZus{}approximation}\PYG{p}{(}\PYG{p}{)}
        
        \PYG{k}{print}\PYG{p}{(}\PYG{l+s+s1}{\PYGZsq{}}\PYG{l+s+s1}{The matrix A:}\PYG{l+s+se}{\PYGZbs{}n}\PYG{l+s+se}{\PYGZbs{}n}\PYG{l+s+s1}{\PYGZsq{}}\PYG{p}{,}\PYG{n}{np}\PYG{o}{.}\PYG{n}{around}\PYG{p}{(}\PYG{n}{rbc}\PYG{o}{.}\PYG{n}{a}\PYG{p}{,}\PYG{l+m+mi}{4}\PYG{p}{)}\PYG{p}{,}\PYG{l+s+s1}{\PYGZsq{}}\PYG{l+s+se}{\PYGZbs{}n}\PYG{l+s+se}{\PYGZbs{}n}\PYG{l+s+s1}{\PYGZsq{}}\PYG{p}{)}
        \PYG{k}{print}\PYG{p}{(}\PYG{l+s+s1}{\PYGZsq{}}\PYG{l+s+s1}{The matrix B:}\PYG{l+s+se}{\PYGZbs{}n}\PYG{l+s+se}{\PYGZbs{}n}\PYG{l+s+s1}{\PYGZsq{}}\PYG{p}{,}\PYG{n}{np}\PYG{o}{.}\PYG{n}{around}\PYG{p}{(}\PYG{n}{rbc}\PYG{o}{.}\PYG{n}{b}\PYG{p}{,}\PYG{l+m+mi}{4}\PYG{p}{)}\PYG{p}{)}
\end{Verbatim}
% This comment is needed to force a line break for adjacent ANSI cells
\begin{OriginalVerbatim}[commandchars=\\\{\}]
The matrix A:

 [[  0.      -0.0083  -0.3599   0.0083   0.    ]
 [  0.       0.       0.       0.       0.    ]
 [  0.      34.3982   0.       0.       0.    ]
 [  0.       0.       0.       0.       0.    ]
 [ -1.       0.       0.       0.       0.    ]]


The matrix B:

 [[ -0.      -0.      -0.3599  -0.      -0.    ]
 [ -3.4497  -1.2074  -0.       3.4497  -0.    ]
 [ -0.      33.5383  -0.      -0.       0.86  ]
 [ -0.      -0.      -2.5898   3.4497  -0.86  ]
 [ -0.9     -0.      -0.      -0.      -0.    ]]
\end{OriginalVerbatim}
Finally, we need to obtain the \sphinxstyleemphasis{solution} to the log-linearized model.
The solution is a pair of matrices \(F\) and \(P\) that specify:
\begin{enumerate}
\item {} 
The current values of the non-state variables \(u_{t}\) as a
linear function of the previous values of the state variables
\(s_t\).

\item {} 
The future values of the state variables \(s_{t+1}\) as a linear
function of the previous values of the state variables \(s_t\)
and the future realisation of the exogenous shock process
\(\epsilon_{t+1}\).

\end{enumerate}
\phantomsection\label{\detokenize{examples:equation-examples:9}}\label{equation:examples:examples:9}\begin{align}
u_t  &  = Fs_t\\
s_{t+1} & = Ps_t + \epsilon_{t+1}.
\end{align}
We use the \sphinxcode{.klein} method to find the solution.

\begin{Verbatim}[commandchars=\\\{\}]
\textcolor{nbsphinxin}{In [9]: }\PYG{c+c1}{\PYGZsh{} Solve the model}
        \PYG{n}{rbc}\PYG{o}{.}\PYG{n}{solve\PYGZus{}klein}\PYG{p}{(}\PYG{n}{rbc}\PYG{o}{.}\PYG{n}{a}\PYG{p}{,}\PYG{n}{rbc}\PYG{o}{.}\PYG{n}{b}\PYG{p}{)}
        
        \PYG{c+c1}{\PYGZsh{} Display the output}
        \PYG{k}{print}\PYG{p}{(}\PYG{l+s+s1}{\PYGZsq{}}\PYG{l+s+s1}{The matrix F:}\PYG{l+s+se}{\PYGZbs{}n}\PYG{l+s+se}{\PYGZbs{}n}\PYG{l+s+s1}{\PYGZsq{}}\PYG{p}{,}\PYG{n}{np}\PYG{o}{.}\PYG{n}{around}\PYG{p}{(}\PYG{n}{rbc}\PYG{o}{.}\PYG{n}{f}\PYG{p}{,}\PYG{l+m+mi}{4}\PYG{p}{)}\PYG{p}{,}\PYG{l+s+s1}{\PYGZsq{}}\PYG{l+s+se}{\PYGZbs{}n}\PYG{l+s+se}{\PYGZbs{}n}\PYG{l+s+s1}{\PYGZsq{}}\PYG{p}{)}
        \PYG{k}{print}\PYG{p}{(}\PYG{l+s+s1}{\PYGZsq{}}\PYG{l+s+s1}{The matrix P:}\PYG{l+s+se}{\PYGZbs{}n}\PYG{l+s+se}{\PYGZbs{}n}\PYG{l+s+s1}{\PYGZsq{}}\PYG{p}{,}\PYG{n}{np}\PYG{o}{.}\PYG{n}{around}\PYG{p}{(}\PYG{n}{rbc}\PYG{o}{.}\PYG{n}{p}\PYG{p}{,}\PYG{l+m+mi}{4}\PYG{p}{)}\PYG{p}{)}
\end{Verbatim}
% This comment is needed to force a line break for adjacent ANSI cells
\begin{OriginalVerbatim}[commandchars=\\\{\}]
The matrix F:

 [[ 0.2297  0.513 ]
 [ 1.      0.35  ]
 [ 3.3197 -0.1408]]


The matrix P:

 [[ 0.9     0.    ]
 [ 0.083   0.9715]]
\end{OriginalVerbatim}

\subsubsection{Impulse responses}
\label{\detokenize{examples:Impulse-responses}}
One the model is solved, use the \sphinxcode{.impulse} method to compute impulse
responses to exogenous shocks to the state. The method creates the
\sphinxcode{.irs} attribute which is a dictionary with keys equal to the names of
the exogenous shocks and the values are Pandas DataFrames with the
computed impulse respones. You can supply your own values for the
shocks, but the default is 0.01 for each exogenous shock.

\begin{Verbatim}[commandchars=\\\{\}]
\textcolor{nbsphinxin}{In [10]: }\PYG{c+c1}{\PYGZsh{} Compute impulse responses and plot}
         \PYG{n}{rbc}\PYG{o}{.}\PYG{n}{impulse}\PYG{p}{(}\PYG{n}{T}\PYG{o}{=}\PYG{l+m+mi}{41}\PYG{p}{,}\PYG{n}{t0}\PYG{o}{=}\PYG{l+m+mi}{1}\PYG{p}{,}\PYG{n}{shock}\PYG{o}{=}\PYG{n+nb+bp}{None}\PYG{p}{,}\PYG{n}{percent}\PYG{o}{=}\PYG{n+nb+bp}{True}\PYG{p}{)}
         
         \PYG{k}{print}\PYG{p}{(}\PYG{l+s+s1}{\PYGZsq{}}\PYG{l+s+s1}{Impulse responses to a 0.01 unit shock to A:}\PYG{l+s+se}{\PYGZbs{}n}\PYG{l+s+se}{\PYGZbs{}n}\PYG{l+s+s1}{\PYGZsq{}}\PYG{p}{,}\PYG{n}{rbc}\PYG{o}{.}\PYG{n}{irs}\PYG{p}{[}\PYG{l+s+s1}{\PYGZsq{}}\PYG{l+s+s1}{eA}\PYG{l+s+s1}{\PYGZsq{}}\PYG{p}{]}\PYG{o}{.}\PYG{n}{head}\PYG{p}{(}\PYG{p}{)}\PYG{p}{)}
\end{Verbatim}
% This comment is needed to force a line break for adjacent ANSI cells
\begin{OriginalVerbatim}[commandchars=\\\{\}]
Impulse responses to a 0.01 unit shock to A:

        a         c   eA         i         k         y
0  0.000  0.000000  0.0  0.000000  0.000000  0.000000
1  1.000  0.229718  1.0  3.319739  0.000000  1.000000
2  0.900  0.249318  0.0  2.976083  0.082993  0.929048
3  0.810  0.265744  0.0  2.667127  0.155321  0.864362
4  0.729  0.279348  0.0  2.389390  0.218116  0.805341
\end{OriginalVerbatim}
Plotting is easy.

\begin{Verbatim}[commandchars=\\\{\}]
\textcolor{nbsphinxin}{In [11]: }\PYG{n}{rbc}\PYG{o}{.}\PYG{n}{irs}\PYG{p}{[}\PYG{l+s+s1}{\PYGZsq{}}\PYG{l+s+s1}{eA}\PYG{l+s+s1}{\PYGZsq{}}\PYG{p}{]}\PYG{p}{[}\PYG{p}{[}\PYG{l+s+s1}{\PYGZsq{}}\PYG{l+s+s1}{a}\PYG{l+s+s1}{\PYGZsq{}}\PYG{p}{,}\PYG{l+s+s1}{\PYGZsq{}}\PYG{l+s+s1}{k}\PYG{l+s+s1}{\PYGZsq{}}\PYG{p}{,}\PYG{l+s+s1}{\PYGZsq{}}\PYG{l+s+s1}{c}\PYG{l+s+s1}{\PYGZsq{}}\PYG{p}{,}\PYG{l+s+s1}{\PYGZsq{}}\PYG{l+s+s1}{y}\PYG{l+s+s1}{\PYGZsq{}}\PYG{p}{,}\PYG{l+s+s1}{\PYGZsq{}}\PYG{l+s+s1}{i}\PYG{l+s+s1}{\PYGZsq{}}\PYG{p}{]}\PYG{p}{]}\PYG{o}{.}\PYG{n}{plot}\PYG{p}{(}\PYG{n}{lw}\PYG{o}{=}\PYG{l+s+s1}{\PYGZsq{}}\PYG{l+s+s1}{5}\PYG{l+s+s1}{\PYGZsq{}}\PYG{p}{,}\PYG{n}{alpha}\PYG{o}{=}\PYG{l+m+mf}{0.5}\PYG{p}{,}\PYG{n}{grid}\PYG{o}{=}\PYG{n+nb+bp}{True}\PYG{p}{)}\PYG{o}{.}\PYG{n}{legend}\PYG{p}{(}\PYG{n}{loc}\PYG{o}{=}\PYG{l+s+s1}{\PYGZsq{}}\PYG{l+s+s1}{upper right}\PYG{l+s+s1}{\PYGZsq{}}\PYG{p}{,}\PYG{n}{ncol}\PYG{o}{=}\PYG{l+m+mi}{2}\PYG{p}{)}
         \PYG{n}{rbc}\PYG{o}{.}\PYG{n}{irs}\PYG{p}{[}\PYG{l+s+s1}{\PYGZsq{}}\PYG{l+s+s1}{eA}\PYG{l+s+s1}{\PYGZsq{}}\PYG{p}{]}\PYG{p}{[}\PYG{p}{[}\PYG{l+s+s1}{\PYGZsq{}}\PYG{l+s+s1}{eA}\PYG{l+s+s1}{\PYGZsq{}}\PYG{p}{,}\PYG{l+s+s1}{\PYGZsq{}}\PYG{l+s+s1}{a}\PYG{l+s+s1}{\PYGZsq{}}\PYG{p}{]}\PYG{p}{]}\PYG{o}{.}\PYG{n}{plot}\PYG{p}{(}\PYG{n}{lw}\PYG{o}{=}\PYG{l+s+s1}{\PYGZsq{}}\PYG{l+s+s1}{5}\PYG{l+s+s1}{\PYGZsq{}}\PYG{p}{,}\PYG{n}{alpha}\PYG{o}{=}\PYG{l+m+mf}{0.5}\PYG{p}{,}\PYG{n}{grid}\PYG{o}{=}\PYG{n+nb+bp}{True}\PYG{p}{)}\PYG{o}{.}\PYG{n}{legend}\PYG{p}{(}\PYG{n}{loc}\PYG{o}{=}\PYG{l+s+s1}{\PYGZsq{}}\PYG{l+s+s1}{upper right}\PYG{l+s+s1}{\PYGZsq{}}\PYG{p}{,}\PYG{n}{ncol}\PYG{o}{=}\PYG{l+m+mi}{2}\PYG{p}{)}
\end{Verbatim}

\begin{Verbatim}[commandchars=\\\{\}]
\textcolor{nbsphinxout}{Out[11]: }\PYGZlt{}matplotlib.legend.Legend at 0x1171f0b70\PYGZgt{}
\end{Verbatim}

\noindent\sphinxincludegraphics{{examples_21_1}.png}

\noindent\sphinxincludegraphics{{examples_21_2}.png}


\subsubsection{Stochastic simulation}
\label{\detokenize{examples:Stochastic-simulation}}
Creating a stochastic simulation of the model is straightforward with
the \sphinxcode{.stoch\_sim} method. In the following example, I create a 151
period (including t=0) simulation by first siumlating the model for 251
periods and then dropping the first 100 values. The variance of the
shock to \(A_t\) is set to 0.001 and the variance of the shock to
\(K_t\) is set to zero because there is not capital shock in the
model. The seed for the numpy random number generator is set to 0.

\begin{Verbatim}[commandchars=\\\{\}]
\textcolor{nbsphinxin}{In [12]: }\PYG{n}{rbc}\PYG{o}{.}\PYG{n}{stoch\PYGZus{}sim}\PYG{p}{(}\PYG{n}{T}\PYG{o}{=}\PYG{l+m+mi}{121}\PYG{p}{,}\PYG{n}{dropFirst}\PYG{o}{=}\PYG{l+m+mi}{100}\PYG{p}{,}\PYG{n}{covMat}\PYG{o}{=}\PYG{n}{np}\PYG{o}{.}\PYG{n}{array}\PYG{p}{(}\PYG{p}{[}\PYG{p}{[}\PYG{l+m+mf}{0.00763}\PYG{o}{*}\PYG{o}{*}\PYG{l+m+mi}{2}\PYG{p}{,}\PYG{l+m+mi}{0}\PYG{p}{]}\PYG{p}{,}\PYG{p}{[}\PYG{l+m+mi}{0}\PYG{p}{,}\PYG{l+m+mi}{0}\PYG{p}{]}\PYG{p}{]}\PYG{p}{)}\PYG{p}{,}\PYG{n}{seed}\PYG{o}{=}\PYG{l+m+mi}{0}\PYG{p}{,}\PYG{n}{percent}\PYG{o}{=}\PYG{n+nb+bp}{True}\PYG{p}{)}
         \PYG{n}{rbc}\PYG{o}{.}\PYG{n}{simulated}\PYG{p}{[}\PYG{p}{[}\PYG{l+s+s1}{\PYGZsq{}}\PYG{l+s+s1}{k}\PYG{l+s+s1}{\PYGZsq{}}\PYG{p}{,}\PYG{l+s+s1}{\PYGZsq{}}\PYG{l+s+s1}{c}\PYG{l+s+s1}{\PYGZsq{}}\PYG{p}{,}\PYG{l+s+s1}{\PYGZsq{}}\PYG{l+s+s1}{y}\PYG{l+s+s1}{\PYGZsq{}}\PYG{p}{,}\PYG{l+s+s1}{\PYGZsq{}}\PYG{l+s+s1}{i}\PYG{l+s+s1}{\PYGZsq{}}\PYG{p}{]}\PYG{p}{]}\PYG{o}{.}\PYG{n}{plot}\PYG{p}{(}\PYG{n}{lw}\PYG{o}{=}\PYG{l+s+s1}{\PYGZsq{}}\PYG{l+s+s1}{5}\PYG{l+s+s1}{\PYGZsq{}}\PYG{p}{,}\PYG{n}{alpha}\PYG{o}{=}\PYG{l+m+mf}{0.5}\PYG{p}{,}\PYG{n}{grid}\PYG{o}{=}\PYG{n+nb+bp}{True}\PYG{p}{)}\PYG{o}{.}\PYG{n}{legend}\PYG{p}{(}\PYG{n}{loc}\PYG{o}{=}\PYG{l+s+s1}{\PYGZsq{}}\PYG{l+s+s1}{upper right}\PYG{l+s+s1}{\PYGZsq{}}\PYG{p}{,}\PYG{n}{ncol}\PYG{o}{=}\PYG{l+m+mi}{4}\PYG{p}{)}
         \PYG{n}{rbc}\PYG{o}{.}\PYG{n}{simulated}\PYG{p}{[}\PYG{p}{[}\PYG{l+s+s1}{\PYGZsq{}}\PYG{l+s+s1}{a}\PYG{l+s+s1}{\PYGZsq{}}\PYG{p}{]}\PYG{p}{]}\PYG{o}{.}\PYG{n}{plot}\PYG{p}{(}\PYG{n}{lw}\PYG{o}{=}\PYG{l+s+s1}{\PYGZsq{}}\PYG{l+s+s1}{5}\PYG{l+s+s1}{\PYGZsq{}}\PYG{p}{,}\PYG{n}{alpha}\PYG{o}{=}\PYG{l+m+mf}{0.5}\PYG{p}{,}\PYG{n}{grid}\PYG{o}{=}\PYG{n+nb+bp}{True}\PYG{p}{)}\PYG{o}{.}\PYG{n}{legend}\PYG{p}{(}\PYG{n}{ncol}\PYG{o}{=}\PYG{l+m+mi}{4}\PYG{p}{)}
         \PYG{n}{rbc}\PYG{o}{.}\PYG{n}{simulated}\PYG{p}{[}\PYG{p}{[}\PYG{l+s+s1}{\PYGZsq{}}\PYG{l+s+s1}{eA}\PYG{l+s+s1}{\PYGZsq{}}\PYG{p}{,}\PYG{l+s+s1}{\PYGZsq{}}\PYG{l+s+s1}{eK}\PYG{l+s+s1}{\PYGZsq{}}\PYG{p}{]}\PYG{p}{]}\PYG{o}{.}\PYG{n}{plot}\PYG{p}{(}\PYG{n}{lw}\PYG{o}{=}\PYG{l+s+s1}{\PYGZsq{}}\PYG{l+s+s1}{5}\PYG{l+s+s1}{\PYGZsq{}}\PYG{p}{,}\PYG{n}{alpha}\PYG{o}{=}\PYG{l+m+mf}{0.5}\PYG{p}{,}\PYG{n}{grid}\PYG{o}{=}\PYG{n+nb+bp}{True}\PYG{p}{)}\PYG{o}{.}\PYG{n}{legend}\PYG{p}{(}\PYG{n}{ncol}\PYG{o}{=}\PYG{l+m+mi}{4}\PYG{p}{)}
\end{Verbatim}

\begin{Verbatim}[commandchars=\\\{\}]
\textcolor{nbsphinxout}{Out[12]: }\PYGZlt{}matplotlib.legend.Legend at 0x117604cc0\PYGZgt{}
\end{Verbatim}

\noindent\sphinxincludegraphics{{examples_23_1}.png}

\noindent\sphinxincludegraphics{{examples_23_2}.png}

\noindent\sphinxincludegraphics{{examples_23_3}.png}


\subsection{Example 3: A New-Keynesian business cycle model}
\label{\detokenize{examples:Example-3:-A-New-Keynesian-business-cycle-model}}
Consider the new-Keynesian model from Walsh (2010), chapter 8 expressed
in log-linear terms:
\phantomsection\label{\detokenize{examples:equation-examples:10}}\label{equation:examples:examples:10}\begin{align}
y_t & = E_ty_{t+1} - \sigma^{-1} (i_t - E_t\pi_{t+1}) + g_t\\
\pi_t & = \beta  E_t\pi_{t+1} + \kappa  y_t + u_t\\
i_t & = \phi_x  y_t + \phi_{\pi}  \pi_t + v_t\\
r_t & = i_t - E_t\pi_{t+1}\\
g_{t+1} & = \rho_g g_{t} + \epsilon_{t+1}^g\\
u_{t+1} & = \rho_u u_{t} + \epsilon_{t+1}^u\\
v_{t+1} & = \rho_v v_{t} + \epsilon_{t+1}^v
\end{align}
where \(y_t\) is the output gap (log-deviation of output from the
natural rate), \(\pi_t\) is the quarterly rate of inflation between
\(t-1\) and \(t\), \(i_t\) is the nominal interest rate on
funds moving between period \(t\) and \(t+1\), \(r_t\) is
the real interest rate, \(g_t\) is the exogenous component of
demand, \(u_t\) is an exogenous component of inflation, and
\(v_t\) is the exogenous component of monetary policy.

Since the model is already log-linear, there is no need to approximate
the equilibrium conditions. We'll still use the \sphinxcode{.log\_linear} method
to find the matrices \(A\) and \(B\), but we'll have to set the
\sphinxcode{islinear} option to \sphinxcode{True} to avoid generating an error.

\begin{Verbatim}[commandchars=\\\{\}]
\textcolor{nbsphinxin}{In [13]: }\PYG{c+c1}{\PYGZsh{} Input model parameters}
         \PYG{n}{beta} \PYG{o}{=} \PYG{l+m+mf}{0.99}
         \PYG{n}{sigma}\PYG{o}{=} \PYG{l+m+mi}{1}
         \PYG{n}{eta}  \PYG{o}{=} \PYG{l+m+mi}{1}
         \PYG{n}{omega}\PYG{o}{=} \PYG{l+m+mf}{0.8}
         \PYG{n}{kappa}\PYG{o}{=} \PYG{p}{(}\PYG{n}{sigma}\PYG{o}{+}\PYG{n}{eta}\PYG{p}{)}\PYG{o}{*}\PYG{p}{(}\PYG{l+m+mi}{1}\PYG{o}{\PYGZhy{}}\PYG{n}{omega}\PYG{p}{)}\PYG{o}{*}\PYG{p}{(}\PYG{l+m+mi}{1}\PYG{o}{\PYGZhy{}}\PYG{n}{beta}\PYG{o}{*}\PYG{n}{omega}\PYG{p}{)}\PYG{o}{/}\PYG{n}{omega}
         
         \PYG{n}{rhor} \PYG{o}{=} \PYG{l+m+mf}{0.9}
         \PYG{n}{phipi}\PYG{o}{=} \PYG{l+m+mf}{1.5}
         \PYG{n}{phiy} \PYG{o}{=} \PYG{l+m+mi}{0}
         
         \PYG{n}{rhog} \PYG{o}{=} \PYG{l+m+mf}{0.5}
         \PYG{n}{rhou} \PYG{o}{=} \PYG{l+m+mf}{0.5}
         \PYG{n}{rhov} \PYG{o}{=} \PYG{l+m+mf}{0.9}
         
         \PYG{n}{Sigma} \PYG{o}{=} \PYG{l+m+mf}{0.001}\PYG{o}{*}\PYG{n}{np}\PYG{o}{.}\PYG{n}{eye}\PYG{p}{(}\PYG{l+m+mi}{3}\PYG{p}{)}
         
         \PYG{c+c1}{\PYGZsh{} This time we\PYGZsq{}ll input the model parameters a list of values and we\PYGZsq{}ll include a list of name strings. And}
         \PYG{c+c1}{\PYGZsh{} the program will consolidate the information into a Pandas Series. FYI the parameters list is required,}
         \PYG{c+c1}{\PYGZsh{} but the parameter names is optional. If omitted, the names are set to \PYGZsq{}parameter 1\PYGZsq{}, \PYGZsq{}parameter 2\PYGZsq{}, etc.}
         
         \PYG{n}{parameters} \PYG{o}{=} \PYG{p}{[}\PYG{n}{beta}\PYG{p}{,}\PYG{n}{sigma}\PYG{p}{,}\PYG{n}{eta}\PYG{p}{,}\PYG{n}{omega}\PYG{p}{,}\PYG{n}{kappa}\PYG{p}{,}\PYG{n}{rhor}\PYG{p}{,}\PYG{n}{phipi}\PYG{p}{,}\PYG{n}{phiy}\PYG{p}{,}\PYG{n}{rhog}\PYG{p}{,}\PYG{n}{rhou}\PYG{p}{,}\PYG{n}{rhov}\PYG{p}{]}
         \PYG{n}{parameterNames} \PYG{o}{=} \PYG{p}{[}\PYG{l+s+s1}{\PYGZsq{}}\PYG{l+s+s1}{beta}\PYG{l+s+s1}{\PYGZsq{}}\PYG{p}{,}\PYG{l+s+s1}{\PYGZsq{}}\PYG{l+s+s1}{sigma}\PYG{l+s+s1}{\PYGZsq{}}\PYG{p}{,}\PYG{l+s+s1}{\PYGZsq{}}\PYG{l+s+s1}{eta}\PYG{l+s+s1}{\PYGZsq{}}\PYG{p}{,}\PYG{l+s+s1}{\PYGZsq{}}\PYG{l+s+s1}{omega}\PYG{l+s+s1}{\PYGZsq{}}\PYG{p}{,}\PYG{l+s+s1}{\PYGZsq{}}\PYG{l+s+s1}{kappa}\PYG{l+s+s1}{\PYGZsq{}}\PYG{p}{,}\PYG{l+s+s1}{\PYGZsq{}}\PYG{l+s+s1}{rhor}\PYG{l+s+s1}{\PYGZsq{}}\PYG{p}{,}\PYG{l+s+s1}{\PYGZsq{}}\PYG{l+s+s1}{phipi}\PYG{l+s+s1}{\PYGZsq{}}\PYG{p}{,}\PYG{l+s+s1}{\PYGZsq{}}\PYG{l+s+s1}{phiy}\PYG{l+s+s1}{\PYGZsq{}}\PYG{p}{,}\PYG{l+s+s1}{\PYGZsq{}}\PYG{l+s+s1}{rhog}\PYG{l+s+s1}{\PYGZsq{}}\PYG{p}{,}\PYG{l+s+s1}{\PYGZsq{}}\PYG{l+s+s1}{rhou}\PYG{l+s+s1}{\PYGZsq{}}\PYG{p}{,}\PYG{l+s+s1}{\PYGZsq{}}\PYG{l+s+s1}{rhov}\PYG{l+s+s1}{\PYGZsq{}}\PYG{p}{]}
         
         \PYG{k}{def} \PYG{n+nf}{equilibrium\PYGZus{}equations}\PYG{p}{(}\PYG{n}{variables\PYGZus{}forward}\PYG{p}{,}\PYG{n}{variables\PYGZus{}current}\PYG{p}{,}\PYG{n}{parameters}\PYG{p}{)}\PYG{p}{:}
         
             \PYG{c+c1}{\PYGZsh{} Parameters}
             \PYG{n}{p} \PYG{o}{=} \PYG{n}{parameters}
         
             \PYG{c+c1}{\PYGZsh{} Variables}
             \PYG{n}{fwd} \PYG{o}{=} \PYG{n}{variables\PYGZus{}forward}
             \PYG{n}{cur} \PYG{o}{=} \PYG{n}{variables\PYGZus{}current}
         
             \PYG{c+c1}{\PYGZsh{} Exogenous demand}
             \PYG{n}{g\PYGZus{}proc} \PYG{o}{=}  \PYG{n}{p}\PYG{o}{.}\PYG{n}{rhog}\PYG{o}{*}\PYG{n}{cur}\PYG{o}{.}\PYG{n}{g} \PYG{o}{\PYGZhy{}} \PYG{n}{fwd}\PYG{o}{.}\PYG{n}{g}
         
             \PYG{c+c1}{\PYGZsh{} Exogenous inflation}
             \PYG{n}{u\PYGZus{}proc} \PYG{o}{=}  \PYG{n}{p}\PYG{o}{.}\PYG{n}{rhou}\PYG{o}{*}\PYG{n}{cur}\PYG{o}{.}\PYG{n}{u} \PYG{o}{\PYGZhy{}} \PYG{n}{fwd}\PYG{o}{.}\PYG{n}{u}
         
             \PYG{c+c1}{\PYGZsh{} Exogenous monetary policy}
             \PYG{n}{v\PYGZus{}proc} \PYG{o}{=}  \PYG{n}{p}\PYG{o}{.}\PYG{n}{rhov}\PYG{o}{*}\PYG{n}{cur}\PYG{o}{.}\PYG{n}{v} \PYG{o}{\PYGZhy{}} \PYG{n}{fwd}\PYG{o}{.}\PYG{n}{v}
         
             \PYG{c+c1}{\PYGZsh{} Euler equation}
             \PYG{n}{euler\PYGZus{}eqn} \PYG{o}{=} \PYG{n}{fwd}\PYG{o}{.}\PYG{n}{y} \PYG{o}{\PYGZhy{}}\PYG{l+m+mi}{1}\PYG{o}{/}\PYG{n}{p}\PYG{o}{.}\PYG{n}{sigma}\PYG{o}{*}\PYG{p}{(}\PYG{n}{cur}\PYG{o}{.}\PYG{n}{i}\PYG{o}{\PYGZhy{}}\PYG{n}{fwd}\PYG{o}{.}\PYG{n}{pi}\PYG{p}{)} \PYG{o}{+} \PYG{n}{fwd}\PYG{o}{.}\PYG{n}{g} \PYG{o}{\PYGZhy{}} \PYG{n}{cur}\PYG{o}{.}\PYG{n}{y}
         
             \PYG{c+c1}{\PYGZsh{} NK Phillips curve evolution}
             \PYG{n}{phillips\PYGZus{}curve} \PYG{o}{=} \PYG{n}{p}\PYG{o}{.}\PYG{n}{beta}\PYG{o}{*}\PYG{n}{fwd}\PYG{o}{.}\PYG{n}{pi} \PYG{o}{+} \PYG{n}{p}\PYG{o}{.}\PYG{n}{kappa}\PYG{o}{*}\PYG{n}{cur}\PYG{o}{.}\PYG{n}{y} \PYG{o}{+} \PYG{n}{fwd}\PYG{o}{.}\PYG{n}{u} \PYG{o}{\PYGZhy{}} \PYG{n}{cur}\PYG{o}{.}\PYG{n}{pi}
         
             \PYG{c+c1}{\PYGZsh{} interest rate rule}
             \PYG{n}{interest\PYGZus{}rule} \PYG{o}{=} \PYG{n}{p}\PYG{o}{.}\PYG{n}{phiy}\PYG{o}{*}\PYG{n}{cur}\PYG{o}{.}\PYG{n}{y}\PYG{o}{+}\PYG{n}{p}\PYG{o}{.}\PYG{n}{phipi}\PYG{o}{*}\PYG{n}{cur}\PYG{o}{.}\PYG{n}{pi} \PYG{o}{+} \PYG{n}{fwd}\PYG{o}{.}\PYG{n}{v} \PYG{o}{\PYGZhy{}} \PYG{n}{cur}\PYG{o}{.}\PYG{n}{i}
         
             \PYG{c+c1}{\PYGZsh{} Fisher equation}
             \PYG{n}{fisher\PYGZus{}eqn} \PYG{o}{=} \PYG{n}{cur}\PYG{o}{.}\PYG{n}{i} \PYG{o}{\PYGZhy{}} \PYG{n}{fwd}\PYG{o}{.}\PYG{n}{pi} \PYG{o}{\PYGZhy{}} \PYG{n}{cur}\PYG{o}{.}\PYG{n}{r}
         
         
             \PYG{c+c1}{\PYGZsh{} Stack equilibrium conditions into a numpy array}
             \PYG{k}{return} \PYG{n}{np}\PYG{o}{.}\PYG{n}{array}\PYG{p}{(}\PYG{p}{[}
                     \PYG{n}{g\PYGZus{}proc}\PYG{p}{,}
                     \PYG{n}{u\PYGZus{}proc}\PYG{p}{,}
                     \PYG{n}{v\PYGZus{}proc}\PYG{p}{,}
                     \PYG{n}{euler\PYGZus{}eqn}\PYG{p}{,}
                     \PYG{n}{phillips\PYGZus{}curve}\PYG{p}{,}
                     \PYG{n}{interest\PYGZus{}rule}\PYG{p}{,}
                     \PYG{n}{fisher\PYGZus{}eqn}
                 \PYG{p}{]}\PYG{p}{)}
         
         \PYG{c+c1}{\PYGZsh{} Initialize the nk}
         \PYG{n}{nk} \PYG{o}{=} \PYG{n}{ls}\PYG{o}{.}\PYG{n}{model}\PYG{p}{(}\PYG{n}{equilibrium\PYGZus{}equations}\PYG{p}{,}
                       \PYG{n}{nstates}\PYG{o}{=}\PYG{l+m+mi}{3}\PYG{p}{,}
                       \PYG{n}{varNames}\PYG{o}{=}\PYG{p}{[}\PYG{l+s+s1}{\PYGZsq{}}\PYG{l+s+s1}{g}\PYG{l+s+s1}{\PYGZsq{}}\PYG{p}{,}\PYG{l+s+s1}{\PYGZsq{}}\PYG{l+s+s1}{u}\PYG{l+s+s1}{\PYGZsq{}}\PYG{p}{,}\PYG{l+s+s1}{\PYGZsq{}}\PYG{l+s+s1}{v}\PYG{l+s+s1}{\PYGZsq{}}\PYG{p}{,}\PYG{l+s+s1}{\PYGZsq{}}\PYG{l+s+s1}{i}\PYG{l+s+s1}{\PYGZsq{}}\PYG{p}{,}\PYG{l+s+s1}{\PYGZsq{}}\PYG{l+s+s1}{r}\PYG{l+s+s1}{\PYGZsq{}}\PYG{p}{,}\PYG{l+s+s1}{\PYGZsq{}}\PYG{l+s+s1}{y}\PYG{l+s+s1}{\PYGZsq{}}\PYG{p}{,}\PYG{l+s+s1}{\PYGZsq{}}\PYG{l+s+s1}{pi}\PYG{l+s+s1}{\PYGZsq{}}\PYG{p}{]}\PYG{p}{,}
                       \PYG{n}{shockNames}\PYG{o}{=}\PYG{p}{[}\PYG{l+s+s1}{\PYGZsq{}}\PYG{l+s+s1}{eG}\PYG{l+s+s1}{\PYGZsq{}}\PYG{p}{,}\PYG{l+s+s1}{\PYGZsq{}}\PYG{l+s+s1}{eU}\PYG{l+s+s1}{\PYGZsq{}}\PYG{p}{,}\PYG{l+s+s1}{\PYGZsq{}}\PYG{l+s+s1}{eV}\PYG{l+s+s1}{\PYGZsq{}}\PYG{p}{]}\PYG{p}{,}
                       \PYG{n}{parameters}\PYG{o}{=}\PYG{n}{parameters}\PYG{p}{,}
                       \PYG{n}{parameterNames}\PYG{o}{=}\PYG{n}{parameterNames}\PYG{p}{)}
         
         \PYG{c+c1}{\PYGZsh{} Set the steady state of the nk}
         \PYG{n}{nk}\PYG{o}{.}\PYG{n}{set\PYGZus{}ss}\PYG{p}{(}\PYG{p}{[}\PYG{l+m+mi}{0}\PYG{p}{,}\PYG{l+m+mi}{0}\PYG{p}{,}\PYG{l+m+mi}{0}\PYG{p}{,}\PYG{l+m+mi}{0}\PYG{p}{,}\PYG{l+m+mi}{0}\PYG{p}{,}\PYG{l+m+mi}{0}\PYG{p}{,}\PYG{l+m+mi}{0}\PYG{p}{]}\PYG{p}{)}
         
         \PYG{c+c1}{\PYGZsh{} Find the log\PYGZhy{}linear approximation around the non\PYGZhy{}stochastic steady state}
         \PYG{n}{nk}\PYG{o}{.}\PYG{n}{log\PYGZus{}linear\PYGZus{}approximation}\PYG{p}{(}\PYG{n}{isloglinear}\PYG{o}{=}\PYG{n+nb+bp}{True}\PYG{p}{)}
         
         \PYG{c+c1}{\PYGZsh{} Solve the nk}
         \PYG{n}{nk}\PYG{o}{.}\PYG{n}{solve\PYGZus{}klein}\PYG{p}{(}\PYG{n}{nk}\PYG{o}{.}\PYG{n}{a}\PYG{p}{,}\PYG{n}{nk}\PYG{o}{.}\PYG{n}{b}\PYG{p}{)}
\end{Verbatim}

\begin{Verbatim}[commandchars=\\\{\}]
\textcolor{nbsphinxin}{In [14]: }\PYG{c+c1}{\PYGZsh{} Compute impulse responses and plot}
         \PYG{n}{nk}\PYG{o}{.}\PYG{n}{impulse}\PYG{p}{(}\PYG{n}{T}\PYG{o}{=}\PYG{l+m+mi}{11}\PYG{p}{,}\PYG{n}{t0}\PYG{o}{=}\PYG{l+m+mi}{1}\PYG{p}{,}\PYG{n}{shock}\PYG{o}{=}\PYG{n+nb+bp}{None}\PYG{p}{)}
         
         \PYG{c+c1}{\PYGZsh{} Create the figure and axes}
         \PYG{n}{fig} \PYG{o}{=} \PYG{n}{plt}\PYG{o}{.}\PYG{n}{figure}\PYG{p}{(}\PYG{n}{figsize}\PYG{o}{=}\PYG{p}{(}\PYG{l+m+mi}{12}\PYG{p}{,}\PYG{l+m+mi}{12}\PYG{p}{)}\PYG{p}{)}
         \PYG{n}{ax1} \PYG{o}{=} \PYG{n}{fig}\PYG{o}{.}\PYG{n}{add\PYGZus{}subplot}\PYG{p}{(}\PYG{l+m+mi}{3}\PYG{p}{,}\PYG{l+m+mi}{1}\PYG{p}{,}\PYG{l+m+mi}{1}\PYG{p}{)}
         \PYG{n}{ax2} \PYG{o}{=} \PYG{n}{fig}\PYG{o}{.}\PYG{n}{add\PYGZus{}subplot}\PYG{p}{(}\PYG{l+m+mi}{3}\PYG{p}{,}\PYG{l+m+mi}{1}\PYG{p}{,}\PYG{l+m+mi}{2}\PYG{p}{)}
         \PYG{n}{ax3} \PYG{o}{=} \PYG{n}{fig}\PYG{o}{.}\PYG{n}{add\PYGZus{}subplot}\PYG{p}{(}\PYG{l+m+mi}{3}\PYG{p}{,}\PYG{l+m+mi}{1}\PYG{p}{,}\PYG{l+m+mi}{3}\PYG{p}{)}
         
         \PYG{c+c1}{\PYGZsh{} Plot commands}
         \PYG{n}{nk}\PYG{o}{.}\PYG{n}{irs}\PYG{p}{[}\PYG{l+s+s1}{\PYGZsq{}}\PYG{l+s+s1}{eG}\PYG{l+s+s1}{\PYGZsq{}}\PYG{p}{]}\PYG{p}{[}\PYG{p}{[}\PYG{l+s+s1}{\PYGZsq{}}\PYG{l+s+s1}{g}\PYG{l+s+s1}{\PYGZsq{}}\PYG{p}{,}\PYG{l+s+s1}{\PYGZsq{}}\PYG{l+s+s1}{y}\PYG{l+s+s1}{\PYGZsq{}}\PYG{p}{,}\PYG{l+s+s1}{\PYGZsq{}}\PYG{l+s+s1}{i}\PYG{l+s+s1}{\PYGZsq{}}\PYG{p}{,}\PYG{l+s+s1}{\PYGZsq{}}\PYG{l+s+s1}{pi}\PYG{l+s+s1}{\PYGZsq{}}\PYG{p}{,}\PYG{l+s+s1}{\PYGZsq{}}\PYG{l+s+s1}{r}\PYG{l+s+s1}{\PYGZsq{}}\PYG{p}{]}\PYG{p}{]}\PYG{o}{.}\PYG{n}{plot}\PYG{p}{(}\PYG{n}{lw}\PYG{o}{=}\PYG{l+s+s1}{\PYGZsq{}}\PYG{l+s+s1}{5}\PYG{l+s+s1}{\PYGZsq{}}\PYG{p}{,}\PYG{n}{alpha}\PYG{o}{=}\PYG{l+m+mf}{0.5}\PYG{p}{,}\PYG{n}{grid}\PYG{o}{=}\PYG{n+nb+bp}{True}\PYG{p}{,}\PYG{n}{title}\PYG{o}{=}\PYG{l+s+s1}{\PYGZsq{}}\PYG{l+s+s1}{Demand shock}\PYG{l+s+s1}{\PYGZsq{}}\PYG{p}{,}\PYG{n}{ax}\PYG{o}{=}\PYG{n}{ax1}\PYG{p}{)}\PYG{o}{.}\PYG{n}{legend}\PYG{p}{(}\PYG{n}{loc}\PYG{o}{=}\PYG{l+s+s1}{\PYGZsq{}}\PYG{l+s+s1}{upper right}\PYG{l+s+s1}{\PYGZsq{}}\PYG{p}{,}\PYG{n}{ncol}\PYG{o}{=}\PYG{l+m+mi}{5}\PYG{p}{)}
         \PYG{n}{nk}\PYG{o}{.}\PYG{n}{irs}\PYG{p}{[}\PYG{l+s+s1}{\PYGZsq{}}\PYG{l+s+s1}{eU}\PYG{l+s+s1}{\PYGZsq{}}\PYG{p}{]}\PYG{p}{[}\PYG{p}{[}\PYG{l+s+s1}{\PYGZsq{}}\PYG{l+s+s1}{u}\PYG{l+s+s1}{\PYGZsq{}}\PYG{p}{,}\PYG{l+s+s1}{\PYGZsq{}}\PYG{l+s+s1}{y}\PYG{l+s+s1}{\PYGZsq{}}\PYG{p}{,}\PYG{l+s+s1}{\PYGZsq{}}\PYG{l+s+s1}{i}\PYG{l+s+s1}{\PYGZsq{}}\PYG{p}{,}\PYG{l+s+s1}{\PYGZsq{}}\PYG{l+s+s1}{pi}\PYG{l+s+s1}{\PYGZsq{}}\PYG{p}{,}\PYG{l+s+s1}{\PYGZsq{}}\PYG{l+s+s1}{r}\PYG{l+s+s1}{\PYGZsq{}}\PYG{p}{]}\PYG{p}{]}\PYG{o}{.}\PYG{n}{plot}\PYG{p}{(}\PYG{n}{lw}\PYG{o}{=}\PYG{l+s+s1}{\PYGZsq{}}\PYG{l+s+s1}{5}\PYG{l+s+s1}{\PYGZsq{}}\PYG{p}{,}\PYG{n}{alpha}\PYG{o}{=}\PYG{l+m+mf}{0.5}\PYG{p}{,}\PYG{n}{grid}\PYG{o}{=}\PYG{n+nb+bp}{True}\PYG{p}{,}\PYG{n}{title}\PYG{o}{=}\PYG{l+s+s1}{\PYGZsq{}}\PYG{l+s+s1}{Inflation shock}\PYG{l+s+s1}{\PYGZsq{}}\PYG{p}{,}\PYG{n}{ax}\PYG{o}{=}\PYG{n}{ax2}\PYG{p}{)}\PYG{o}{.}\PYG{n}{legend}\PYG{p}{(}\PYG{n}{loc}\PYG{o}{=}\PYG{l+s+s1}{\PYGZsq{}}\PYG{l+s+s1}{upper right}\PYG{l+s+s1}{\PYGZsq{}}\PYG{p}{,}\PYG{n}{ncol}\PYG{o}{=}\PYG{l+m+mi}{5}\PYG{p}{)}
         \PYG{n}{nk}\PYG{o}{.}\PYG{n}{irs}\PYG{p}{[}\PYG{l+s+s1}{\PYGZsq{}}\PYG{l+s+s1}{eV}\PYG{l+s+s1}{\PYGZsq{}}\PYG{p}{]}\PYG{p}{[}\PYG{p}{[}\PYG{l+s+s1}{\PYGZsq{}}\PYG{l+s+s1}{v}\PYG{l+s+s1}{\PYGZsq{}}\PYG{p}{,}\PYG{l+s+s1}{\PYGZsq{}}\PYG{l+s+s1}{y}\PYG{l+s+s1}{\PYGZsq{}}\PYG{p}{,}\PYG{l+s+s1}{\PYGZsq{}}\PYG{l+s+s1}{i}\PYG{l+s+s1}{\PYGZsq{}}\PYG{p}{,}\PYG{l+s+s1}{\PYGZsq{}}\PYG{l+s+s1}{pi}\PYG{l+s+s1}{\PYGZsq{}}\PYG{p}{,}\PYG{l+s+s1}{\PYGZsq{}}\PYG{l+s+s1}{r}\PYG{l+s+s1}{\PYGZsq{}}\PYG{p}{]}\PYG{p}{]}\PYG{o}{.}\PYG{n}{plot}\PYG{p}{(}\PYG{n}{lw}\PYG{o}{=}\PYG{l+s+s1}{\PYGZsq{}}\PYG{l+s+s1}{5}\PYG{l+s+s1}{\PYGZsq{}}\PYG{p}{,}\PYG{n}{alpha}\PYG{o}{=}\PYG{l+m+mf}{0.5}\PYG{p}{,}\PYG{n}{grid}\PYG{o}{=}\PYG{n+nb+bp}{True}\PYG{p}{,}\PYG{n}{title}\PYG{o}{=}\PYG{l+s+s1}{\PYGZsq{}}\PYG{l+s+s1}{Interest rate shock}\PYG{l+s+s1}{\PYGZsq{}}\PYG{p}{,}\PYG{n}{ax}\PYG{o}{=}\PYG{n}{ax3}\PYG{p}{)}\PYG{o}{.}\PYG{n}{legend}\PYG{p}{(}\PYG{n}{loc}\PYG{o}{=}\PYG{l+s+s1}{\PYGZsq{}}\PYG{l+s+s1}{upper right}\PYG{l+s+s1}{\PYGZsq{}}\PYG{p}{,}\PYG{n}{ncol}\PYG{o}{=}\PYG{l+m+mi}{5}\PYG{p}{)}
\end{Verbatim}

\begin{Verbatim}[commandchars=\\\{\}]
\textcolor{nbsphinxout}{Out[14]: }\PYGZlt{}matplotlib.legend.Legend at 0x117ae4ef0\PYGZgt{}
\end{Verbatim}

\noindent\sphinxincludegraphics{{examples_26_1}.png}

\begin{Verbatim}[commandchars=\\\{\}]
\textcolor{nbsphinxin}{In [15]: }\PYG{n}{nk}\PYG{o}{.}\PYG{n}{stoch\PYGZus{}sim}\PYG{p}{(}\PYG{n}{T}\PYG{o}{=}\PYG{l+m+mi}{151}\PYG{p}{,}\PYG{n}{dropFirst}\PYG{o}{=}\PYG{l+m+mi}{100}\PYG{p}{,}\PYG{n}{covMat}\PYG{o}{=}\PYG{n}{Sigma}\PYG{p}{,}\PYG{n}{seed}\PYG{o}{=}\PYG{l+m+mi}{0}\PYG{p}{)}
         
         \PYG{c+c1}{\PYGZsh{} Create the figure and axes}
         \PYG{n}{fig} \PYG{o}{=} \PYG{n}{plt}\PYG{o}{.}\PYG{n}{figure}\PYG{p}{(}\PYG{n}{figsize}\PYG{o}{=}\PYG{p}{(}\PYG{l+m+mi}{12}\PYG{p}{,}\PYG{l+m+mi}{8}\PYG{p}{)}\PYG{p}{)}
         \PYG{n}{ax1} \PYG{o}{=} \PYG{n}{fig}\PYG{o}{.}\PYG{n}{add\PYGZus{}subplot}\PYG{p}{(}\PYG{l+m+mi}{2}\PYG{p}{,}\PYG{l+m+mi}{1}\PYG{p}{,}\PYG{l+m+mi}{1}\PYG{p}{)}
         \PYG{n}{ax2} \PYG{o}{=} \PYG{n}{fig}\PYG{o}{.}\PYG{n}{add\PYGZus{}subplot}\PYG{p}{(}\PYG{l+m+mi}{2}\PYG{p}{,}\PYG{l+m+mi}{1}\PYG{p}{,}\PYG{l+m+mi}{2}\PYG{p}{)}
         
         \PYG{c+c1}{\PYGZsh{} Plot commands}
         \PYG{n}{nk}\PYG{o}{.}\PYG{n}{simulated}\PYG{p}{[}\PYG{p}{[}\PYG{l+s+s1}{\PYGZsq{}}\PYG{l+s+s1}{y}\PYG{l+s+s1}{\PYGZsq{}}\PYG{p}{,}\PYG{l+s+s1}{\PYGZsq{}}\PYG{l+s+s1}{i}\PYG{l+s+s1}{\PYGZsq{}}\PYG{p}{,}\PYG{l+s+s1}{\PYGZsq{}}\PYG{l+s+s1}{pi}\PYG{l+s+s1}{\PYGZsq{}}\PYG{p}{,}\PYG{l+s+s1}{\PYGZsq{}}\PYG{l+s+s1}{r}\PYG{l+s+s1}{\PYGZsq{}}\PYG{p}{]}\PYG{p}{]}\PYG{o}{.}\PYG{n}{plot}\PYG{p}{(}\PYG{n}{lw}\PYG{o}{=}\PYG{l+s+s1}{\PYGZsq{}}\PYG{l+s+s1}{5}\PYG{l+s+s1}{\PYGZsq{}}\PYG{p}{,}\PYG{n}{alpha}\PYG{o}{=}\PYG{l+m+mf}{0.5}\PYG{p}{,}\PYG{n}{grid}\PYG{o}{=}\PYG{n+nb+bp}{True}\PYG{p}{,}\PYG{n}{title}\PYG{o}{=}\PYG{l+s+s1}{\PYGZsq{}}\PYG{l+s+s1}{Output, inflation, and interest rates}\PYG{l+s+s1}{\PYGZsq{}}\PYG{p}{,}\PYG{n}{ax}\PYG{o}{=}\PYG{n}{ax1}\PYG{p}{)}\PYG{o}{.}\PYG{n}{legend}\PYG{p}{(}\PYG{n}{ncol}\PYG{o}{=}\PYG{l+m+mi}{4}\PYG{p}{)}
         \PYG{n}{nk}\PYG{o}{.}\PYG{n}{simulated}\PYG{p}{[}\PYG{p}{[}\PYG{l+s+s1}{\PYGZsq{}}\PYG{l+s+s1}{g}\PYG{l+s+s1}{\PYGZsq{}}\PYG{p}{,}\PYG{l+s+s1}{\PYGZsq{}}\PYG{l+s+s1}{u}\PYG{l+s+s1}{\PYGZsq{}}\PYG{p}{,}\PYG{l+s+s1}{\PYGZsq{}}\PYG{l+s+s1}{v}\PYG{l+s+s1}{\PYGZsq{}}\PYG{p}{]}\PYG{p}{]}\PYG{o}{.}\PYG{n}{plot}\PYG{p}{(}\PYG{n}{lw}\PYG{o}{=}\PYG{l+s+s1}{\PYGZsq{}}\PYG{l+s+s1}{5}\PYG{l+s+s1}{\PYGZsq{}}\PYG{p}{,}\PYG{n}{alpha}\PYG{o}{=}\PYG{l+m+mf}{0.5}\PYG{p}{,}\PYG{n}{grid}\PYG{o}{=}\PYG{n+nb+bp}{True}\PYG{p}{,}\PYG{n}{title}\PYG{o}{=}\PYG{l+s+s1}{\PYGZsq{}}\PYG{l+s+s1}{Exogenous demand, inflation, and policy}\PYG{l+s+s1}{\PYGZsq{}}\PYG{p}{,}\PYG{n}{ax}\PYG{o}{=}\PYG{n}{ax2}\PYG{p}{)}\PYG{o}{.}\PYG{n}{legend}\PYG{p}{(}\PYG{n}{ncol}\PYG{o}{=}\PYG{l+m+mi}{4}\PYG{p}{,}\PYG{n}{loc}\PYG{o}{=}\PYG{l+s+s1}{\PYGZsq{}}\PYG{l+s+s1}{lower right}\PYG{l+s+s1}{\PYGZsq{}}\PYG{p}{)}
\end{Verbatim}

\begin{Verbatim}[commandchars=\\\{\}]
\textcolor{nbsphinxout}{Out[15]: }\PYGZlt{}matplotlib.legend.Legend at 0x11762ff28\PYGZgt{}
\end{Verbatim}

\noindent\sphinxincludegraphics{{examples_27_1}.png}

\begin{Verbatim}[commandchars=\\\{\}]
\textcolor{nbsphinxin}{In [16]: }\PYG{n}{nk}\PYG{o}{.}\PYG{n}{simulated}\PYG{p}{[}\PYG{p}{[}\PYG{l+s+s1}{\PYGZsq{}}\PYG{l+s+s1}{eG}\PYG{l+s+s1}{\PYGZsq{}}\PYG{p}{,}\PYG{l+s+s1}{\PYGZsq{}}\PYG{l+s+s1}{g}\PYG{l+s+s1}{\PYGZsq{}}\PYG{p}{]}\PYG{p}{]}\PYG{o}{.}\PYG{n}{plot}\PYG{p}{(}\PYG{n}{lw}\PYG{o}{=}\PYG{l+s+s1}{\PYGZsq{}}\PYG{l+s+s1}{5}\PYG{l+s+s1}{\PYGZsq{}}\PYG{p}{,}\PYG{n}{alpha}\PYG{o}{=}\PYG{l+m+mf}{0.5}\PYG{p}{,}\PYG{n}{grid}\PYG{o}{=}\PYG{n+nb+bp}{True}\PYG{p}{)}\PYG{o}{.}\PYG{n}{legend}\PYG{p}{(}\PYG{n}{ncol}\PYG{o}{=}\PYG{l+m+mi}{2}\PYG{p}{)}
         \PYG{n}{nk}\PYG{o}{.}\PYG{n}{simulated}\PYG{p}{[}\PYG{p}{[}\PYG{l+s+s1}{\PYGZsq{}}\PYG{l+s+s1}{eU}\PYG{l+s+s1}{\PYGZsq{}}\PYG{p}{,}\PYG{l+s+s1}{\PYGZsq{}}\PYG{l+s+s1}{u}\PYG{l+s+s1}{\PYGZsq{}}\PYG{p}{]}\PYG{p}{]}\PYG{o}{.}\PYG{n}{plot}\PYG{p}{(}\PYG{n}{lw}\PYG{o}{=}\PYG{l+s+s1}{\PYGZsq{}}\PYG{l+s+s1}{5}\PYG{l+s+s1}{\PYGZsq{}}\PYG{p}{,}\PYG{n}{alpha}\PYG{o}{=}\PYG{l+m+mf}{0.5}\PYG{p}{,}\PYG{n}{grid}\PYG{o}{=}\PYG{n+nb+bp}{True}\PYG{p}{)}\PYG{o}{.}\PYG{n}{legend}\PYG{p}{(}\PYG{n}{ncol}\PYG{o}{=}\PYG{l+m+mi}{2}\PYG{p}{)}
         \PYG{n}{nk}\PYG{o}{.}\PYG{n}{simulated}\PYG{p}{[}\PYG{p}{[}\PYG{l+s+s1}{\PYGZsq{}}\PYG{l+s+s1}{eV}\PYG{l+s+s1}{\PYGZsq{}}\PYG{p}{,}\PYG{l+s+s1}{\PYGZsq{}}\PYG{l+s+s1}{v}\PYG{l+s+s1}{\PYGZsq{}}\PYG{p}{]}\PYG{p}{]}\PYG{o}{.}\PYG{n}{plot}\PYG{p}{(}\PYG{n}{lw}\PYG{o}{=}\PYG{l+s+s1}{\PYGZsq{}}\PYG{l+s+s1}{5}\PYG{l+s+s1}{\PYGZsq{}}\PYG{p}{,}\PYG{n}{alpha}\PYG{o}{=}\PYG{l+m+mf}{0.5}\PYG{p}{,}\PYG{n}{grid}\PYG{o}{=}\PYG{n+nb+bp}{True}\PYG{p}{)}\PYG{o}{.}\PYG{n}{legend}\PYG{p}{(}\PYG{n}{ncol}\PYG{o}{=}\PYG{l+m+mi}{2}\PYG{p}{)}
\end{Verbatim}

\begin{Verbatim}[commandchars=\\\{\}]
\textcolor{nbsphinxout}{Out[16]: }\PYGZlt{}matplotlib.legend.Legend at 0x117a49be0\PYGZgt{}
\end{Verbatim}

\noindent\sphinxincludegraphics{{examples_28_1}.png}

\noindent\sphinxincludegraphics{{examples_28_2}.png}

\noindent\sphinxincludegraphics{{examples_28_3}.png}


\section{How \sphinxstyleliteralintitle{linearsolve} works}
\label{\detokenize{howLinearsolveWorks::doc}}\label{\detokenize{howLinearsolveWorks:How-linearsolve-works}}

\subsection{The \sphinxstyleliteralintitle{linearsolve.model} class}
\label{\detokenize{howLinearsolveWorks:The-linearsolve.model-class}}
The equilibrium conditions for most DSGE models can be expressed as a vector function \(F\):
\phantomsection\label{\detokenize{howLinearsolveWorks:equation-howLinearsolveWorks:0}}\begin{equation}\label{equation:howLinearsolveWorks:howLinearsolveWorks:0}
\begin{split}f(E_t X_{t+1}, X_t, \epsilon_{t+1}) = 0,\end{split}
\end{equation}
where 0 is an \(n\times 1\) vector of zeros, \(X_t\) is an \(n\times 1\) vector of endogenous variables, and \(\epsilon_{t+1}\) is an \(m\times 1\) vector of exogenous structural shocks to the model. \(E_tX_{t+1}\) denotes the expecation of the \(t+1\) endogenous variables based on the information available to decision makers in the model as of time period \(t\).

The function \(f\) is often nonlinear. Because the values of the endogenous variables in period \(t\) depend on the expected future values of those variables, it is not in general possible to compute the equilibirum of the model by working directly with the function \(f\). Instead it is often convenient to work with a log-linear approximation to the equilibrium conditions around a non-stochastic steady state. In many cases, the log-linear approximation can be written in the following form:
\phantomsection\label{\detokenize{howLinearsolveWorks:equation-howLinearsolveWorks:1}}\begin{equation}\label{equation:howLinearsolveWorks:howLinearsolveWorks:1}
\begin{split}A E_t\left[ x_{t+1} \right] & = B x_t + \left[ \begin{array}{c} \epsilon_{t+1} \\ 0 \end{array} \right],\end{split}
\end{equation}
where the vector \(x_{t}\) denotes the log deviation of the variables in \(X_t\) from their steady state values. The variables in \(x_t\) are grouped in a specific way: \(x_t = [s_t; u_t]\) where \(s_t\) is an \(n_s \times 1\) vector of predetermined (state) variables and \(u_t\) is an \(n_u \times 1\) vector of nonpredetermined (forward-looking) variables. \(\epsilon_{t+1}\) is an \(n_s\times 1\) vector of i.i.d. shocks to the state variables \(s_{t+1}\).  \(\epsilon_{t+1}\) has mean 0 and diagonal covariance matrix \(\Sigma\). The solution to the model is a pair of matrices \(F\) and \(P\) such that:
\phantomsection\label{\detokenize{howLinearsolveWorks:equation-howLinearsolveWorks:2}}\begin{equation}\label{equation:howLinearsolveWorks:howLinearsolveWorks:2}
\begin{split}s_{t+1} &= Ps_t + \epsilon_{t+1}\\
u_t &= Fs_t,\end{split}
\end{equation}
and:
\phantomsection\label{\detokenize{howLinearsolveWorks:equation-howLinearsolveWorks:3}}\begin{equation}\label{equation:howLinearsolveWorks:howLinearsolveWorks:3}
\begin{split}s_{t+1} = Ps_t + \epsilon_{t+1}.\end{split}
\end{equation}
The matrices \(F\) and \(P\) are obtained using the \href{http://www.sciencedirect.com/science/article/pii/S0165188999000457}{Klein (2000)}. solution method which is based on the generalized Schur factorization of the marices \(A\) and \(B\). The solution routine incorporates many aspects of his program for Matlab \href{http://paulklein.ca/newsite/codes/codes.php}{solab.m}.

This package defines a \sphinxtitleref{linearsolve.model} class. An instance of the \sphinxtitleref{linearsolve.model} has the following methods:
\begin{enumerate}
\item {} 
\sphinxcode{compute\_ss(guess,method,options)}: Computes the steady state of the nonlinear model.

\item {} 
\sphinxcode{set\_ss(steady\_state)}: Sets the steady state \sphinxcode{.ss} attribute of the instance.

\item {} 
\sphinxcode{log\_linear\_approximation(steady\_state,isloglinear)}: Log-linearizes the nonlinear model and constructs the matrices \(A\) and \(B\).

\item {} 
\sphinxcode{klein(a,b)}: Solves the linear model using Klein's solution method.

\item {} 
\sphinxcode{approximate\_and\_solve(isloglinear)}: Approximates and solves the model by combining the previous two methods.

\item {} 
\sphinxcode{impulse(T,t0,shock,percent)}: Computes impulse responses for shocks to each endogenous state variable.

\item {} 
\sphinxcode{approximated(round,precision)}: Returns a string containing the log-linear approximation to the equilibrium conditions of the model.

\item {} 
\sphinxcode{solved(round,precision)}: Returns a string containing the solution to the log-linear approximation of the model.

\end{enumerate}

In this notebook, I demonstrate how to use the module to simulate two basic business cycle models: an real business cycle (RBC) model and a new-Keynesian business cycle model.

\begin{Verbatim}[commandchars=\\\{\}]
\textcolor{nbsphinxin}{In [ ]: }
\end{Verbatim}

\begin{Verbatim}[commandchars=\\\{\}]
\textcolor{nbsphinxin}{In [ ]: }
\end{Verbatim}


\chapter{Indices and tables}
\label{\detokenize{index:indices-and-tables}}\begin{itemize}
\item {} 
\DUrole{xref,std,std-ref}{genindex}

\item {} 
\DUrole{xref,std,std-ref}{search}

\end{itemize}



\renewcommand{\indexname}{Index}
\printindex
\end{document}